Im Folgenden werden die Ergebnisse der Auswertung diskutiert. Es ist zu erwähnen, dass nicht jede Schaltung untersucht werden konnte, da bereits bei der Signalgenerator-Schaltung keine Änderung am Ausgangssignal nachgewiesen konnte und nach mehreren Behebungs-Versuchen diese und folgende Messungen abgebrochen wurde.

\subsection{Linearverstärker}
Die qualitativ erwartete Verstärkungskurve konnte nur bei zwei der drei aufgebauten Konfigurationen beobachtet werden, hat aber hier Werte angenommen die um Faktoren von $2\sim 3$ von der Theorie abweichen. Bei der dritten Konfiguration hat die Verstärkungskurve erst sehr viel später als erwartet abgenommen. Wir vermuten hier schlechten Kontakt mit dem Steckboard, sodass der OPV nicht korrekt verbunden war.
Die gemessene Phasenverschiebung zeigt einen Verlauf der mit den Verstärkungskurve übereinstimmt, aber weist im Verstärkungsbereich ein Phasendifferenz nahe null statt der erwarteten $180$° auf. Dies kann durch lange Leitungen, problematischen Verbindungen mit dem OPV oder fehlerhaften OPVs erklärt werden.

\subsection{Integrator und Differenzierer}
Die qualitative Integration kann bei allen Wellenfunktionen gut beobachtet werden. Die qualitative Verstärkungskurve entspricht jedoch nicht der Theorie. Sehr auffällig ist hierbei die starke Abweichung beim Neu-Aufbau an einer anderen Stelle auf dem Steckboard, was nahelegt das hier große Probleme in den Kontakten des Steckboards vorliegen
Beim Differenzierer sind stark verrauschte Bilder zu beobachten, die Probleme in den Kontakten der Schaltung oder in den OPV nahelegen. Die gemessenen Frequenzabhängigkeiten beider Schaltungen weichen ebenfalls stark von der Theorie ab, was ebenfalls fehlerhafte Verbindungen mit dem OPV nahelegt.

\subsection{Schmitt-Trigger}
Die Schwellwerte des aufgebauten Schmitt-Triggers konnten mittels des Oszilloskops klar erkannt werden, weichen aber um mehr als eine Zehnerpotenz von der Theorie ab. Hier können wieder fehlerhafte OPVs oder schlechte Verbindungen im Steckboard die Ursache sein.

\subsection{Signalgenerator}
Für die Frequenz des Ausgangssignals stehen keine Parameter bereit, mit denen sich die
theoretische Frequenz $\nu$ nach Gleichung \ref{eqn:10} berechnen lässt. Daher
kann auch keine Einschätzung über die Qualität des angegeben Wertes der Frequenz
angegeben werden. Die Form des Signals entspricht einem einseitigen Flankensignal,
bei welchem lediglich für die aufsteigende Flanke eine Signaländerung erreicht wird.

\subsection{Zusammenfassung}
Bei den meisten Schaltungen können die qualitativen Ideen der Schaltungen in mindestens beschränktem Maße bestätigt werden, die tatsächlichen Verstärkungswerte weichen jedoch bei jeder Konfiguration stark von der Theorie ab. Hier sind Probleme in dem Aufbau der Schaltungen zu vermuten.
Diese Probleme können von Problemen im Steckboard, instabiler Versorgungsspannungen bis zu fehlerhaften OPVs reichen. Bei den abgebrochen Teilen des Versuchs wurden ebenfalls Probleme mit der Eingangsspannung beobachtet, die in den ersten Schaltungen nicht am Oszilloskop beobachtet wurden.
