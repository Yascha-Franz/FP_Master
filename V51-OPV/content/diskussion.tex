Im Folgenden werden die Ergebnisse der Auswertung diskutiert. Es ist zu erwähnen, dass nicht jede Schaltung untersucht werden konnte, da bereits bei der Signalgenerator-Schaltung keine Änderung am Ausgangssignal nachgewiesen konnte und nach mehreren Behebungs-Versuchen diese und folgende Messungen abgebrochen wurde.

\subsection{Linearverstärker}
Die qualitativ erwartete Verstärkungskurve konnte nur bei zwei der drei aufgebauten Konfigurationen beobachtet werden, hat aber hier Werte angenommen die um Faktoren von $2\sim 3$ von der Theorie abweichen. Die gemessene Phasenverschiebung zeigt einen Verlauf der mit den Verstärkungskurve übereinstimmt, aber weist im Verstärkungsbereich ein Phasendifferenz nahe null statt der erwarteten $180$° auf. Dies kann durch lange Leitungen oder fehlerhafte Operationsverstärker erklärt werden.

\subsection{Integrator und Differenzierer}
Die qualitative Integration und Differenzierung können bei allem Wellenfunktionen außer der differenzierenden Dreieckswelle gut bis beschränkt gut beobachtet werden. Es ist aber beim Differenzierer stark verrauschte Bilder zu beobachten, die Probleme in den Kontakten der Schaltung oder in den Operationsverstärkern nahelegen. Die gemessenen Frequenzabhängigkeiten beider Schaltungen weichen ebenfalls stark von der Theorie ab, was ebenfalls fehlerhafte Operationsverstärker nahelegt.

\subsection{Schmitt-Trigger}
Die Schwellwerte des aufgebauten Schmitt-Triggers konnten mittels des Oszilloskops klar erkannt werden, weichen aber um mehr als eine Zehnerpotenz von der Theorie ab. Dies kann ebenfalls wieder mit Problemen in den Operationsverstärkern erklärt werden.

\subsection{Zusammenfassung}
Bei den meisten Schaltungen können die qualitativen Ideen der Schaltungen in mindestens beschränktem Maße bestätigt werden, die tatsächlichen Verstärkungswerte weichen jedoch bei jeder Konfiguration stark von der Theorie ab. Hier sind Probleme in dem Aufbau der Schaltungen zu vermuten. Bei manchen der Schaltungen könnten Kontaktprobleme im Steckboard reichen um die Messfehler zu erklären, das wahrscheinlichere Problem sind aber nicht funktionsfähige Operationsverstärker.