\noindent Die \enquote{nuclear magnetic resonance}, kurz NMR, beschreibt Phänomene, welche
bei der Wechselwirkung zwischen elektromagnetischer Strahlung und Materie in
Erscheinung treten. Die Energieniveaus, welche sich für verschiedene magnetische
Felder der Flussdichte $B$ ergeben, stehen dabei in direkter Verbindung zu den
Orientierungen der magnetischen Momente der Atome im angelegten magnetischen
Feld \cite{nmr_relaxation}.
\subsection{Kern- und Elektronenmagnetismus}
\noindent Magnetisch wechselwirkende Kerne besitzen neben einem magnetischen
Moment einen Drehimpuls $\vec{J}$. Dieser ist proportional zum magnetischen
Moment $\vec{\mu} = \gamma \cdot \vec{J}$, wobei der Proportionalitätsfaktor
$\gamma$ den gyromagnetischen Faktor beschreibt. Unter Einfluss eines äußeren
magnetischen Feldes mit einer Flussdichte $B$ koppeln das magnetische Moment und das magnetische Feld
zu einem Drehmoment $\vec{D} = \vec{\mu} \times B = \frac{d \vec{J}}{dt}$.
Für konstante, homogene Felder mit fester räumlicher Orientierung
(beispielsweise in z-Richtung) präzediert das magnetische Moment $\vec{\mu}$
in Richtung des magnetischen Feldes. Die zur Präzession gehörige
Frequenz ist dabei die Lamorfrequenz $\nu = \frac{\omega}{2 \cdot \pi} =
\frac{\gamma \cdot B_z}{2 \cdot \pi}$. Der Index $z$ der magnetischen Feldstärke
gibt dabei an, in welche Richtung (hier z-Richtung) das Feld
ausgerichtet ist. Die Frequenz $\omega$ entspricht der zur Lamorfrequenz gehörigen
Winkelgeschwindigkeit.
\subsection{Magnetisierung und Relaxation}
\noindent Um Effekte der Kernspinresonanz untersuchen zu können, ist ein
statisches Magnetfeld $B_\text{stat.}$ notwendig. Zusätzlich befindet sich die Probe in
einer Spule, in welche durch weitere, kurze Strompulse magnetische Felder $B_\text{puls.}$ induziert
werden, welche unabhängig vom statischen Feld sind. Die Pulse lassen sich dazu nutzen, das thermische Gleichgewicht der
Probe zu stören und die Magnetisierung um einen Winkel proportional zur
Pulsdauer zu drehen. \\
\noindent Nach Abklingen des kurzen Strompulses und Verschwinden des
Magnetfeldes $B_\text{puls.}$ beginnt die angeregte Magnetisierung der Probe ins thermische
Gleichgewicht zurück zu relaxieren. Das Relaxiationsverhalten wird dabei durch
die Bloch-Gleichungen
\begin{align*}
  \left(\frac{dM_x}{dt} \right) &= \left(-\frac{M_x}{T_2} \right),  \\
  \left(\frac{dM_y}{dt} \right) &= \left(-\frac{M_y}{T_2} \right) 
\end{align*}
\begin{align}
  \left(\frac{dM_z}{dt} \right) &= \left(\frac{M_0 - M_z}{T_1} \right)
  % M_z = M_0 \cdot \left(1 - \exp{\left(-\frac{t}{T_1}\right)}\right)
  \label{eqn:01}
\end{align}
\noindent beschrieben. Der Index $i$ gibt auch hier wieder die Richtung der
Magnetisierung an, $M_0$ beschreibt die Magnetisierung im thermischen
Gleichgewicht, $T_1$ beschreibt die Spin-Gitter-Relaxationszeit und $T_2$
entspricht der Spin-Spin-Relaxationszeit. \\
\subsubsection{Spin-Gitter-Relaxationszeit}
Die Relaxationszeit in z-Richtung wird in den Bloch Gleichungen 1 bis 2
durch $T_1$ beschrieben. Diese Zeitkonstante beschreibt dabei die Eigenschaften
des Spinsystems sowie die Eigenschaften dessen unmittelbarer Umgebung. Für große
Werte von $T_1$ besteht lediglich eine schwache Kopplung von Spinsystem und
Umgebung, kleine Werte lassen auf eine starke Kopplung beider Systeme schließen.
Während der Relaxation wird die potentielle Energie der Kernspins der Probe in
eine Gitterschwingung umgewandelt. \\
\subsubsection{Spin-Spin-Relaxation}
\noindent Während Spin-Spin-Relaxation auf die z-Richtung beschränkt ist, findet
die Spin-Spin-Relaxation in der xy-Ebene statt. In dieser Ebene resultiert die
Magnetisierung aus allen Präzessionsbewegungen um die z-Achse. Da die Magnetisierung nicht
volkommen homogen über die Ebene verteilt ist, unterscheiden sich die zu den
Präzessionen gehörigen Lamor-Frequenzen gering voneinander, sodass die
Spins bereits nach kurzer Zeit nicht weiter in Phase schwingen \cite{nmr}.
Explizit äußert sich die Dephasierung dabei auf zwei verschiedenen Arten. Zum
einen sorgt die inhomogene Verteilung für eine statische Feldinhomogenität, deren
Reversibilität im Hahn-Echo-Verfahren genutzt wird, um die Feldinhomogenität zu
minimieren und $T_2$ bestimmen zu können. Zum anderen zeigt sich eine irreversible
Dekohärenz, angeregt beispielsweise durch Stoßprozesse oder eine $T_1$-Relaxation,
welche Einfluss auf den $T_2$-Parameter nimmt.
\subsection{Diffusion}
\noindent Für sphärische Moleküle mit dem Radius $r$ liefert die
Einstein-Stokes Gleichung
\begin{align}
  D = \frac{k_B \cdot T}{6 \cdot \pi \cdot \eta \cdot r}
  \label{eqn:02}
\end{align}
\noindent eine Möglichkeit, aus der Diffusionskonstanten $D$ den Molekülradius
$r$ zu bestimmen. Dieser ist Abhängig von der Temperatur $T$ der Moleküle sowie
der Viskosität $\eta$. Die Boltzmann-Konstante $k_B$ drückt ebenso wie der
Faktor $\pi$ lediglich eine Proportionalität aus \cite{stokes}. \\
\noindent In einem magnetischen Gradientenfeld führt die Verschiebung der Spins
aufgrund der Diffusionsprozesse während der $T_2$ Relaxation zu einer
zusätzlichen Signalabnahme. Mit der Diffusions-Zeitkonstante
\begin{align}
  T_D = \frac{3}{\gamma^2 \cdot \tau^2 \cdot G^2 \cdot D}
  \label{diffusion_constant}
\end{align}
lässt sich schließlich
der Diffusionskoeffizient der Einstein-Stokes Gleichung \ref{eqn:02} bestimmen,
welcher sich in einen Molekülradius $r$ umrechnen lässt. Der Faktor $\gamma$
beschreibt dabei noch immer das gyromagnetische Verhältnis, $\tau$ entspricht
dem zeitlichen Abstand der Pulse und $G$ stellt den magnetischen Feldgradienten
dar
% , welcher sich aus $G = 2 \cdot \pi \cdot d_f \cdot(\gamma \cdot d)^{-1}$
% berechnen lässt. Der Durchmesser des gemessenen Spektrums $d_f$ ist dabei in
% $\si{\hertz}$ angegeben, $d$ bezeichnet den Innendurchmesser der Probe
\cite{sample}.
\noindent Unter dem \enquote{freien Induktionsfall} (FID) ist die Relaxation
der Magnetisierung zurück ins thermische Gleichgewicht bekannt, nachdem ein
A-Puls für eine Verkippung innerhalb der xy-Ebene gesorgt hat. Das
Relaxationsverhalten lässt sich dabei beschreiben durch die Gleichungen
\begin{align}
  M_x(t) &= M_0 \cdot \cos{(\omega_0 t)} \cdot \exp{\left( - \frac{t}{T_2}\right)} \qquad \text{und} \\
  M_y(t) &= - M_0 \cdot \sin{(\omega_0 t)} \cdot \exp{\left( - \frac{t}{T_2}\right)}.
\end{align}
\noindent Das sich periodisch ändernde Magnetfeld bewirkt dadurch nach den
Lenz'schen Regeln einen Induktionsstrom.
