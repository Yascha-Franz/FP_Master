\noindent Die \enquote{nuclear magnetic resonance}, kurz NMR beschreibt Phänomene, welche
bei der Wechselwirkung zwischen elektromagnetischer Strahlung mit Materie in
Erscheinung treten. Die Energieniveaus, die bei diesen Wechselwirkungen
auftreten, stehen dabei in direkter Verbindung zu den verschiedenen
Orientierungen der magnetischen Momente der Atome im angelegten magnetischen
Feld \cite{nmr_relaxation}.
\subsection{Kern- und Elektronenmagnetismus}
\noindent Magnetisch wechselwirkende Kerne besitzen neben einem magnetischen
Moment einen Drehimpuls $\vec{J}$. Dieser ist proportional zum magnetischen
Moment
\begin{align}
  \vec{\mu} = \gamma \cdot \vec{J},
  \label{eqn:01}
\end{align}
\noindent wobei der Proportionalitätsfaktor $\gamma$ den gyromagnetischen Faktor
beschreibt. Unter Einfluss eines äußeren magnetischen Feldes $B$ koppeln das
magnetische Moment und das magnetische Feld $B$ zu einem Drehmoment
\begin{align}
  \vec{D} = \vec{\mu} \times B = \frac{d \vec{J}}{dt}.
  \label{eqn:02}
\end{align}
\noindent Für konstante, homogene Felder mit fester räumlicher Orientierung
(beispielsweise in z-Richtung) präzediert das magnetische Moment $\vec{\mu}$
entlang der Richtung des magnetischen Feldes. Die zur Präzession gehörige
Frequenz ist dabei die Lamorfrequenz
\begin{align}
  \nu = \frac{\omega}{2 \cdot \pi} = \frac{\gamma \cdot \B_i}{2 \cdot \pi}.
  \label{eqn:03}
\end{align}
\noindent Der Index $i$ der magnetischen Feldstärke gibt dabei an, in welche
Richtung (x-, y-, oder z-Richtung) das Feld ausgerichtet ist. $\omega$ entspricht
der zur Lamorfrequenz gehörigen Winkelgeschwindigkeit.  
