\noindent Um die Gültigkeit der Messung zu überprüfen, ist es abschließend
erforderlich, die berechneten Relaxationszeiten mit den theoretischen Werten
zu vergleichen.
\subsection{Spin-Gitter-Relaxationszeit}
Die berechnete Spin-Gitter-Relaxationszeit liegt bei
\begin{align*}
  T_1 =  (2.470  \pm  0.040) \: \si{\second},
\end{align*}
\noindent der Literaturwert der Spin-Gitter-Relaxation liegt nach
\cite{temperatur_t1} bei
\begin{align*}
  T_{1,\text{lit.}} = 3.15 \: \si{\second}.
\end{align*}
\noindent Die Abweichung liegt demnach bei etwa $\SI{21.6}{\percent}$.
\subsection{Spin-Spin-Relaxationszeit und Diffusionskonstante}
\noindent Für die Spin-Spin-Relaxationszeit lassen die Messwerte auf einen Wert
\begin{align*}
  T_2 =  (1.822  \pm  0.120) \si{\second} \: \si{\second}
\end{align*}
\noindent schließen. Dieser wird dazu genutzt, die Diffusionskonstante
\begin{align*}
  D =  (2.116 \pm 0.037)e-09 \frac{\si{\meter^2}}{\si{\second}}
\end{align*}
\noindent zu bestimmen. Laut Literatur \cite{diffusion} liegt dieser Wert bei
\begin{align*}
  D_\text{lit.} =  2.299e-09  \frac{\si{\meter^2}}{\si{\second}},
\end{align*}
\noindent die Abweichung beträgt damit etwa $\SI{8}{\percent}$.
\subsection{Molekülradius}
Der aus der Diffusionskonstanten ermittelte Molekülradius beträgt etwa
\begin{align*}
 r = (1.017\pm 0.019) \text{\r{A}},
\end{align*}
\noindent sein geschätzter Wert liegt aber bei
\begin{align*}
 r = 1.742 \text{\r{A}},
\end{align*}
\noindent die Abweichung liegt daher bei etwa $\SI{40.5}{\percent}$. Dies ist den unterschiedlich gewählten Modellen zuzuschreiben. Das Wasser-Molekül ist keineswegs perfekt sphärisch. \\
\noindent Die Gründe für die Messabweichungen sind vielen verschiedenen
Bereichen zuzuordnen. Neben systematischen Ablesefehlern am Oszilloskop konnte
die Viskosität des Wassers nicht exakt zur Temperatur bestimmt werden. Die
Temperaturmessung selbst fand lediglich am Ort der Probe statt, die Temperatur
der Probe selbst wurde nicht gemessen.
