\noindent Zu Beginn des Versuchs wird die Temperatur am Ort der Probe mittels
des Thermometers gemessen. Anschließend wird eine Referenzprobe aus mit
Kupfersulfat versetzten Wasser in die Spule im Permanentmagneten eingesetzt und
die Justage vorgenommen. Für die Startparameter sind dabei die Frequenz $f =
\SI{21.7}{\mega\hertz}$, die Pulslänge des A-Pulses $t_A =
\SI{2}{\micro\second}$, die Anzahl der B-Pulse $N_B = 0$, sowie die Periodendauer
$P = \SI{0.5}{\second}$ und die Shims-Parameter $x = -0.1, y = -5.0, z = 3.7$
und $z^2 = -2.4$ am Steuergerät einzustellen. Bei der Justage der Frequenz ist
zu beachten, dass die Detektion von Real- und Imaginärteil relativ zur
Lamorfrequenz geschieht und dass die Frequenz dann korrekt justiert ist, wenn
keine Schwingungen des Signals mehr messbar sind. Um einen deutlichen
Signaleingang zu erhalten, ist die Phase so zu wählen, dass der Imaginärteil
verschwindet und das Signal im Realteil angetragen wird. Durch die
Shimen-Parameter wird die Feldhomogenität der Spule homogenisiert, die Pulslänge
für einen $\alpha = \SI{90}{\degree}$ Puls (A-Puls) betrage
$\SI{2}{\micro\second}$, für einen $\alpha = \SI{180}{\degree}$ Pulses (B-Puls)
betrage die Pulsdauer $\SI{4}{\micro\second}$. \\
Zur Messung der Spin-Gitter-Relaxationszeit $T_1$ werden die Pulslängen des
A-Pulses und des B-Pulses auf die vorab definierten Parameter eingestellt.
Die Anzahl der B-Pulse werde auf $N_B = 1$ begrenzt. Für die Periode ist eine
Dauer von $P = \SI{10}{\second}$ zu wählen, eine Nachjustage ist erforderlich,
sobald der Abstand zwischen den Pulsen $\tau$ mehr als eine Sekunde beträgt. In
diesem Fall ist eine Periodendauer $P + \tau$ zu wählen. Variiert werden soll in
der Messung der Pulsabstand $\tau$, während die Amplitude des FID nach dem
A-Puls gemessen wird. \\
\noindent Unter dem \enquote{freien Induktionsfall} (FID) ist die Relaxation
der Magnetisierung zurück ins thermische Gleichgewicht bekannt, nachdem ein
A-Puls für eine Verkippung innerhalb der xy-Ebene gesorgt hat. Das
Relaxationsverhalten lässt sich dabei beschreiben durch die Gleichungen
\begin{align}
  M_x(\tau) &= M_0 \cdot \cos{\omega_0 \tau} \cdot \exp{\left( - \frac{\tau}{T_2}\right)} \qquad \text{und} \\
  M_y(\tau) &= - M_0 \cdot \sin{\omega_0 \tau} \cdot \exp{\left( - \frac{\tau}{T_2}\right)}.
\end{align}
\noindent Das sich periodisch ändernde Magnetfeld bewirkt dadurch nach den
Lenz'schen Regeln einen Induktionsstrom. Diese entstehende Schwingung des
Induktionsstromes klingt mit der Zeit ab und wird gemessen. Bei der Messung ist
darauf zu achten, dass $\tau$ so gewählt wird, dass für den kürzesten Abstand
noch keine Relaxation stattgefunden hat, die Magnetisierung der Probe für den
längsten Abstand allerdings vollständig relaxiert ist. Des Weiteren ist die
Messung bis weit über den Nulldurchgang hinaus durchzuführen. 
