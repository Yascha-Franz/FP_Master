Aus der Medizin bekannt und in vielen Bereichen der Physik und anderen
Naturwissenschaften genutzt ist das zerstörungsfreie Messverfahren der Kernspinresonanz.
Ein Anwendungsbereich liegt darin, mittels gepulster Kernspinresonanz verschiedene
Materialeigenschaften wie beispielsweise die Größe von Molekülen zu bestimmen. \\
In diesem Versuch sollen dazu die Spin-Gitter-Relaxation und die
Spin-Spin-Relaxation genauer betrachtet werden. Explizit betrachtet werden die
charakteristischen Relaxationszeiten und die zugehörigen
Diffusionskoeffizienten von Wasser. Dazu wird die magnetische Resonanz an
Protonen genutzt.
