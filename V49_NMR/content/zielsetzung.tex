Aus der Medizin bekannt und in vielen Bereichen der Physik und anderen
Naturwissenschaften ist das zerstörungsfreie Messverfahren der Kernspinresonanz.
Ortsabhängige Vorgänge lassen sich mithilfe des Phänomens der Kernspinresonanz
unter Beobachtung magnetischer Gradientenfelder bildgebend darstellen. \\
In diesem Versuch sollen dazu die Spin-Gitter-Relaxation und die
Spin-Spin-Relaxation genauer betrachtet werden. Explizit betrachtet werden
sollen dazu die charakteristischen Relaxationszeiten und die zugehörigen
Diffusionskoeffizienten von Wasser. Dazu wird die magnetische Resonanz an
Protonen genutzt.  
