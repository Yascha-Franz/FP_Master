\noindent Für die Fehlerrechung wird die empirische Standardabweichung
\begin{equation}
  \sigma = \sqrt{\frac{1}{n-1} \cdot \sum_{i=1}^n(x_i-\overline{x})^2}
  \label{eqn:Stdabweichung}
\end{equation}
und die Gaußsche Fehlerfortpflanzung
\begin{equation}
  u_y = \sqrt{\sum_{i=1}^n\left(\frac{\delta y}{\delta x_i}u_x\right)^2}
  \label{eqn:gauß}
\end{equation}
verwendet.
\noindent Um die Spin-Spin-Relaxationszeit $T_2$ zu bestimmen, werden
aufeinander verschiedene Verfahren angewendet. Beim Hahn-Echo-Verfahren werden
Feldinhomogenitäten durch Ausnutzung der FID kompensiert. Ein A-Puls regt den
FID an, ein B-Puls spiegelt die Auslenkung des Signals an der y-Achse.
Divergierende Spins konvergieren dadurch wieder und es entsteht ein Echo. Die
Messung der Amplituden erfolgt dabei in einem Abstand von $2 \cdot \tau$. Eine
Erweiterung dieses Messverfahrens ist das Carr-Purcell-Messverfahren. In diesem
wird anstelle von einem Signalecho im Abstand $2 \cdot \tau$ eine Vielzahl
dieser Echosignale betrachtet. Problematisch dabei ist allerdings, dass eine
nur gering von der theoretischen Pulslänge abweichende tatsächliche
Pulslänge nicht gänzlich zur Spiegelung des Signals an der y-Achse führt und
so eine mit jedem Schritt ansteigende Messunsicherheit verursacht. Stehen die
B-Pulse allerdings mit einem Versatz von $\Delta \phi = \SI{90}{degree}$ zu den
A-Pulsen, so löschen sich die entstehenden Unsicherheiten bei jeder Spiegelung
selbst aus, für Gesamtpulslängen ganzzahliger Vielfacher von $4 \cdot \tau$
werden korrigierte Werte gemessen. Dieses Verfahren wird auch
Meiboom-Gill-Verfahren genannt.  
