\noindent Für die Fehlerrechung wird die empirische Standardabweichung
\begin{equation}
  \sigma = \sqrt{\frac{1}{n-1} \cdot \sum_{i=1}^n(x_i-\overline{x})^2}
  \label{eqn:Stdabweichung}
\end{equation}
und die Gaußsche Fehlerfortpflanzung
\begin{equation}
  u_y = \sqrt{\sum_{i=1}^n\left(\frac{\delta y}{\delta x_i}u_x\right)^2}
  \label{eqn:gauß}
\end{equation}
verwendet.
\subsection{Messung der Spin-Gitter Relaxationszeit}
\noindent Die Spin-Gitter Relaxationszeit berechnet sich aus den zu den
jeweiligen Pulsabständen $\tau$ gemessenen Spannungen, welche in Tabelle
\ref{table:01} angegeben sind gemäß der Fit-Funktion
\begin{align}
  U(\tau) = M_0 \cdot \exp{\left(- \frac{\tau}{T_1}\right)} + M_1.
  \label{eqn:03}
\end{align}
\noindent Abbildung \ref{fig:01} zeigt die graphische Darstellung der Messwerte
inklusive Ausgleichsgerade. Die zu den längsten Pulsabständen gehörigen Werte
wurden dabei nicht berücksichtigt, um eine bessere Fit-Funktion zu erreichen.
Aus den aufgenommenen Werten lassen sich die Fitparameter
\begin{align*}
  M_0 &=  -(2.026  \pm  0.054) \: \si{\volt} \\
  M_1 &=  (1.045 \pm  0.055) \: \si{\volt} \\
  T_1 &=  (2.609  \pm  0.134) \: \si{\second}
\end{align*}
\noindent bestimmen.
\FloatBarrier
\begin{table}
  \centering
  \caption{Polarisationsabhängigkeit der Intensitätsextrema.}
  \label{tabular_01}
  \begin{tabular}{c c c c | c c c c}
    \toprule
   \multicolumn{1}{c}{$\Theta$} & \multicolumn{1}{c}{$U_\text{min}$} & \multicolumn{1}{c}{$U_\text{max}$}
    & \multicolumn{1}{c}{$K$} & \multicolumn{1}{c}{$\Theta$} & \multicolumn{1}{c}{$U_\text{min}$}
    & \multicolumn{1}{c}{$U_\text{max}$} & \multicolumn{1}{c}{$K$}\\
   \midrule
   \SI{-15}{\degree} & \SI{694}{\milli\volt} & \SI{1710}{\milli\volt} &  {0.42} & \SI{105}{\degree} & \SI{640}{\milli\volt} & \SI{1572}{\milli\volt} &  {0.42} \\
   \SI{0  }{\degree} & \SI{765}{\milli\volt} & \SI{906 }{\milli\volt} &  {0.08} & \SI{120}{\degree} & \SI{306}{\milli\volt} & \SI{2250}{\milli\volt} &  {0.76} \\
   \SI{15 }{\degree} & \SI{377}{\milli\volt} & \SI{712 }{\milli\volt} &  {0.31} & \SI{130}{\degree} & \SI{200}{\milli\volt} & \SI{2670}{\milli\volt} &  {0.86} \\
   \SI{30 }{\degree} & \SI{148}{\milli\volt} & \SI{620 }{\milli\volt} &  {0.61} & \SI{135}{\degree} & \SI{168}{\milli\volt} & \SI{2540}{\milli\volt} &  {0.88} \\
   \SI{40 }{\degree} & \SI{67 }{\milli\volt} & \SI{696 }{\milli\volt} &  {0.82} & \SI{140}{\degree} & \SI{219}{\milli\volt} & \SI{2640}{\milli\volt} &  {0.85} \\
   \SI{45 }{\degree} & \SI{63 }{\milli\volt} & \SI{740 }{\milli\volt} &  {0.84} & \SI{150}{\degree} & \SI{355}{\milli\volt} & \SI{2270}{\milli\volt} &  {0.73} \\
   \SI{50 }{\degree} & \SI{58 }{\milli\volt} & \SI{770 }{\milli\volt} &  {0.86} & \SI{165}{\degree} & \SI{571}{\milli\volt} & \SI{1514}{\milli\volt} &  {0.45} \\
   \SI{60 }{\degree} & \SI{91 }{\milli\volt} & \SI{951 }{\milli\volt} &  {0.83} & \SI{180}{\degree} & \SI{721}{\milli\volt} & \SI{862 }{\milli\volt} &  {0.09} \\
   \SI{75 }{\degree} & \SI{339}{\milli\volt} & \SI{1167}{\milli\volt} &  {0.55} & \SI{195}{\degree} & \SI{317}{\milli\volt} & \SI{704 }{\milli\volt} &  {0.38} \\
   \SI{90 }{\degree} & \SI{809}{\milli\volt} & \SI{1074}{\milli\volt} &  {0.14} \\
\bottomrule
  \end{tabular}
\end{table}

\FloatBarrier
\FloatBarrier
\begin{figure}
  \includegraphics{build/T1.pdf}
  \caption{Spannungsamplituden in Abhängigkeit vom Pulsabstand zur Bestimmung
           der Spin-Gitter-Relaxationszeit $T_1$.}
  \label{fig:01}
\end{figure}
\FloatBarrier
\noindent Die Spin-Gitter-Relaxationszeit liegt damit bei etwa
\begin{align*}
  T_1 &=  (2.609  \pm  0.134) \: \si{\second}.
\end{align*}
\subsection{Messung der Spin-Spin Relaxationszeit}
\noindent analog zu Gleichung \ref{eqn:03} lässt sich eine Gleichung
\begin{align}
  U(\tau) = M_0 \cdot \exp{\left(- \frac{\tau}{T_2}\right)} + M_1.
  \label{eqn:04}
\end{align}
\noindent aufstellen, um aus den gemessenen Daten die
Spin-Spin-Relaxationszeit $T_2$ zu bestimmen. Der verwendete Datensatz stammt
wie in der Durchführung beschrieben aus der $T_2$-Messung mit \enquote{MG on}.
Die Messkurve inklusive Ausgleichskurve ist in Abbildung \ref{fig:02}
dargestellt. Für die Parameter der Fit-Funktion ergeben sich die Werte
\begin{align*}
  M_0 &=  (0.707  \pm  0.020) \si{\volt} \\
  M_1 &=  (0.037  \pm  0.024) \si{\volt} \\
  T_2 &=  (1.822  \pm  0.120) \si{\second}.
\end{align*}
% \noindent Die Dauer einer Spin-Spin-Relaxation beträgt daher
% \begin{align*}
%     T_2 &=  (1.822  \pm  0.120) \si{\second}.
% \end{align*}
\FloatBarrier
\begin{figure}
  \includegraphics{build/T2.pdf}
  \caption{Spannungsamplituden in Abhängigkeit vom Pulsabstand zur Bestimmung
           der Spin-Spin-Relaxationszeit $T_2$.}
  \label{fig:02}
\end{figure}
\FloatBarrier
\subsection{Berechnung der Diffusionskonstanten $D$}
\noindent Zur Berechnung der Diffusionskonstanten $D$ ist es erforderlich, die
Gradientenfeldstärke zu kennen. Zur Berechnung dieser wird der Probendurchmesser
$d = \SI{4.4}{\milli\meter}$ mit einer Frequenz von $d_f = \SI{15200}{\hertz}$, die aus der Fouriertransformation in Abbildung \ref{fig:d_f} abgelesen wird, verwendet. 
\begin{figure}
  \includegraphics{build/echo_gradient.pdf}
  \caption{Stark herangezoomte Fouriertransformation des Echos. Es wird hier nur der Peak der Fouriertransformation betrachtet und daraus der Frequenzdurchmesser $d_f = \SI{15200}{\hertz}$ bestimmt.}
  \label{fig:d_f}
\end{figure}
Über die Formel
\begin{align}
  G = \frac{2 \cdot \pi \cdot d_f}{\gamma \cdot d}
  \label{eqn:05}
\end{align}
\noindent berechnet sich die Gradientenfeldstärke unter Beachtung des
gyromagnetischen Verhältnisses eines Protons $\gamma \approx \num{268e6} \:
\text{rad}(\si{\second\tesla})^{-1}$ zu
$G = \num{0.081} {\si{\tesla}}{\si{\meter}^{-1}}$. \\
\noindent Um daraus die Diffusionskonstante $D$ berechnen zu können, ist die
Relaxationszeit $T_D$ zu berechnen. Die Messungen der über mehrere
Größenordnungen variierenden Pulsabstände lassen sich durch eine Funktion
\begin{align}
  M(\tau) = M_0 \cdot \exp{\left(- \frac{2 \cdot \tau}{T_2}\right)} \cdot \exp{\left(- \frac{2 \cdot \tau^3}{a}\right)} + M_1
 \label{eqn:06}
\end{align}
\noindent approximieren. In Abbildung \ref{fig:03} ist dieser Zusammenhang
graphisch veranschaulicht. Dabei ergeben sich die Parameter
\begin{align*}
  M_0 &=  (0.890  \pm  0.005)          \si{\volt} \\
  M_1 &=  (0.0390  \pm  0.005)         \si{\volt} \\
  a &=  (3.175e-06  \pm  5.792e-08) \si{\second^3}.
\end{align*}
\FloatBarrier
\begin{figure}
  \includegraphics{build/tau_D.pdf}
  \caption{Messwerte der Amplituden in Abhängigkeit vom Pulsabstand $\tau$ und
  approximierte Messkurve zur Bestimmung der Diffusionskonstanten $D$.}
  \label{fig:03}
\end{figure}
\FloatBarrier
\noindent Die verwendeten Messwerte lassen sich Tabelle \ref{table:02} entnehmen.
\FloatBarrier
\begin{table}
  \centering
  \caption{Anzahl der Intensitätsextrema je Drehung der Doppelglashalter um $\alpha = \SI{10}{\degree}$.}
  \label{tabular_02}
  \begin{tabular}{c c | c c}
    \toprule
   \multicolumn{1}{c}{$t_\text{Durchgang}$} & \multicolumn{1}{c}{$n_\text{counts}$} & \multicolumn{1}{c}{$t_\text{Durchgang}$} & \multicolumn{1}{c}{$n_\text{counts}$}\\
   \midrule
    \num{1} & \num{37} & \num{6 } & \num{37} \\
    \num{2} & \num{39} & \num{7 } & \num{35} \\
    \num{3} & \num{36} & \num{8 } & \num{37} \\
    \num{4} & \num{34} & \num{9 } & \num{35} \\
    \num{5} & \num{36} & \num{10} & \num{37} \\
\bottomrule
  \end{tabular}
\end{table}

\FloatBarrier
\noindent Die Relaxationszeit $T_D$ lässt sich dann berechnen über
$T_D = \frac{a}{\tau^3}$, um daraus schließlich Rückschlüsse auf die
Diffusionskonstante zu generieren. Es gilt
\begin{align*}
  D = (2.006e-09 \pm 3.660e-11) \frac{\si{\meter^2}}{\si{\second}}.
\end{align*}
\noindent Aus der Einstein-Stokes-Gleichung \ref{eqn:02} kann daraus unter Verwendung der
Parameter $T = \SI{293.75}{\kelvin}$ (gemessen) und $\eta = \SI{10e-3}{\pascal\second}$
\cite{Viskosität_Wasser} der Molekülradius zu
\begin{align*}
 r = (1.073\pm 1.957e{-2}) \text{\r{A}}
\end{align*}
\noindent bestimmen. Als Vergleichswert lässt sich aus der Gleichung
\begin{align}
  r = \sqrt[3]{\frac{3 \cdot M}{\rho \cdot N_A \cdot 4 \cdot \pi}}
  \label{eqn:07}
\end{align}
\noindent nach \cite{radius} bestimmen. Gleichung \ref{eqn:07} lässt sich dabei
aus dem Volumen der Wassermoleköle herleiten. Approximiert durch eine sphärische
Symmetrie lässt sich das Volumen der Molekülkugel über die molekulare
Dichte der Teilchen $M = \SI{18.015}{\gram\per\mol}$
und deren Dichte $\rho = 1 \: \frac{\si{\gram}}{\si{\cubic\centi\meter}}$
\cite{Viskosität_Wasser}, sowie durch die Avogadro-Konstante $N_A$ bestimmen.
Es ergibt sich ein Schätzwert für den Radius von
\begin{align*}
  r_\text{Schätzung} = \num{1.742} \text{\r{A}}.
\end{align*}
\noindent Die Temperaturmessungen am Ort der Probe beginnen mit einem Wert
$T_{T_1} = \SI{19.5}{\degree}$ während der $T_1$- Messung,
$T_{T_2} = \SI{20.4}{\degree}$ während der $T_2$- Messung und Enden bei dem Wert
$T_{T_D} = \SI{20.6}{\degree}$ während der Messung der Diffusionskonstanten.
