\noindent Für die Fehlerrechung wird die empirische Standardabweichung
\begin{equation}
  \sigma = \sqrt{\frac{1}{n-1} \cdot \sum_{i=1}^n(x_i-\overline{x})^2}
  \label{eqn:Stdabweichung}
\end{equation}
und die Gaußsche Fehlerfortpflanzung
\begin{equation}
  u_y = \sqrt{\sum_{i=1}^n\left(\frac{\delta y}{\delta x_i}u_x\right)^2}
  \label{eqn:gauß}
\end{equation}
verwendet.
\noindent Um die Spin-Spin-Relaxationszeit $T_2$ zu bestimmen, werden
aufeinander verschiedene Verfahren angewendet. Beim Hahn-Echo-Verfahren werden
Feldinhomogenitäten durch Ausnutzung der FID kompensiert. Ein A-Puls regt den
FID an, ein B-Puls spiegelt die Auslenkung des Signals an der y-Achse.
Divergierende Spins konvergieren dadurch wieder und es entsteht ein Echo. Die
Messung der Amplituden erfolgt dabei in einem Abstand von $2 \cdot \tau$. Eine
Erweiterung dieses Messverfahrens ist das Carr-Purcell-Messverfahren. In diesem
wird anstelle von einem Signalecho im Abstand $2 \cdot \tau$ eine Vielzahl
dieser Echosignale betrachtet. Problematisch dabei ist allerdings, dass eine
nur gering von der theoretischen Pulslänge abweichende tatsächliche
Pulslänge nicht gänzlich zur Spiegelung des Signals an der y-Achse führt und
so eine mit jedem Schritt ansteigende Messunsicherheit verursacht. Stehen die
B-Pulse allerdings mit einem Versatz von $\Delta \phi = \SI{90}{degree}$ zu den
A-Pulsen, so löschen sich die entstehenden Unsicherheiten bei jeder Spiegelung
selbst aus, für Gesamtpulslängen ganzzahliger Vielfacher von $4 \cdot \tau$
werden korrigierte Werte gemessen. Dieses Verfahren wird auch
Meiboom-Gill-Verfahren genannt. \\
\subsection{Messung der Spin-Gitter Relaxationszeit}
\noindent Die Spin-Gitter Relaxationszeit berechnet sich aus den zu den
jeweiligen Pulsabständen $\tau$ gemessenen Spannungen, welche in Tabelle
\ref{table:01} angegeben sind gemäß der Fit-Funktion
\begin{align}
  U(\tau) = M_0 \cdot \exp{\left(- \frac{\tau}{T_1}\right)} + M_1.
  \label{eqn:03}
\end{align}
\noindent Abbildung \ref{fig:01} zeigt die graphische Darstellung der Messwerte
inklusive Ausgleichsgerade. Die zu den längsten Pulsabständen gehörigen Werte
wurden dabei nicht berücksichtigt, um eine bessere Fit-Funktion zu erreichen.
Aus den aufgenommenen Werten lassen sich die Fitparameter
\begin{align*}
  M_0 &=  -(2.026  \pm  0.054) \: \si{\volt} \\
  M_1 &=  (1.045 \pm  0.055) \: \si{\volt} \\
  T_1 &=  (2.609  \pm  0.134) \: \si{\second}
\end{align*}
\noindent bestimmen.
\FloatBarrier
\begin{table}
\centering
\caption{Gemessene Zählraten für verschiedene Verzögerungszeiten $t_\text{Verz.}$ bei einer fest gewählten
Referenzzeit $t_\text{Ref.} = \SI{10}{\nano\second}$.}
\sisetup{table-format=3.0}
\begin{tabular}{c c c | c c c}
\toprule
\multicolumn{1}{c}{$t_\text{Verz.} \:/\: \si{\nano\second}$} & \multicolumn{1}{c}{$n_\text{links}$}
& \multicolumn{1}{c}{$n_\text{rechts}$} & \multicolumn{1}{c}{$t_\text{Verz.} \:/\: \si{\nano\second}$}
& \multicolumn{1}{c}{$n_\text{links}$} & \multicolumn{1}{c}{$n_\text{rechts}$}  \\
\midrule
0   & 238 & 238 & 6   & 173 & 168 \\
1   & 239 & 218 & 8   & 116 & 125 \\
1.5 & 243 & 243 & 10  & 89  & 89  \\
2   & 219 & 230 & 12  & 43  & 41  \\
4   & 215 & 200 \\
\bottomrule
\end{tabular}
\label{tabular_01}
\end{table}

\FloatBarrier
\FloatBarrier
\begin{figure}
  \includegraphics{build/T1.pdf}
  \caption{Spannungsamplituden in Abhängigkeit vom Pulsabstand zur Bestimmung
           der Spin-Gitter-Relaxationszeit $T_1$.}
  \label{fig:01}
\end{figure}
\FloatBarrier
\noindent Die Spin-Gitter-Relaxationszeit liegt damit bei etwa
\begin{align*}
  T_1 &=  (2.609  \pm  0.134) \: \si{\second}.
\end{align*}
\subsection{Messung der Spin-Spin Relaxationszeit}
\noindent analog zu Gleichung \ref{eqn:03} lässt sich eine Gleichung
\begin{align}
  U(\tau) = M_0 \cdot \exp{\left(- \frac{\tau}{T_2}\right)} + M_1.
  \label{eqn:04}
\end{align}
\noindent aufstellen, um aus den gemessenen Daten die
Spin-Spin-Relaxationszeit $T_2$ zu bestimmen. Der verwendete Datensatz stammt
wie in der Durchführung beschrieben aus der $T_2$-Messung mit \enquote{MG on}.
Die Messkurve inklusive Ausgleichskurve ist in Abbildung \ref{fig:02}
dargestellt. Für die Parameter der Fit-Funktion ergeben sich die Werte
\begin{align*}
  M_0 &=  -(0.707  \pm  0.020) \si{\volt} \\
  M_1 &=  -(0.037  \pm  0.024) \si{\volt} \\
  T_2 &=  (1.822  \pm  0.120) \si{\second}.
\end{align*}
\noindent Die Dauer einer Spin-Spin-Relaxation beträgt daher
\begin{align*}
    T_2 &=  (1.822  \pm  0.120) \si{\second}.
\end{align*}
\FloatBarrier
\begin{figure}
  \includegraphics{build/T2.pdf}
  \caption{Spannungsamplituden in Abhängigkeit vom Pulsabstand zur Bestimmung
           der Spin-Gitter-Relaxationszeit $T_2$.}
  \label{fig:02}
\end{figure}
\FloatBarrier
\subsection{Berechnung der Diffusionskonstanten $D$}
\noindent Zur Berechnung der Diffusionskonstanten $D$ ist es erforderlich, die
Gradientenfeldstärke zu kennen. Zur Berechnung dieser wird der Probendurchmesser
$d = \SI{4.4}{\milli\meter}$ und der Probendurchmesser ausgedrückt durch eine
Frequenz zu $d_f = \SI{15200}{\hertz}$ aus der Fouriertransformation in Abbildung \ref{fig:d_f} abgelesen. 
\begin{figure}
  \includegraphics{build/echo_gradient.pdf}
  \caption{Stark herangezoomte Fouriertransformation des Echos. Es wird hier nur der Peak der Fouriertransformation betrachtet und daraus der Frequenzdurchmesser $d_f = \SI{15200}{\hertz}$ bestimmt.}
  \label{fig:d_f}
\end{figure}
Über die Formel
\begin{align}
  G = \frac{2 \cdot \pi \cdot d_f}{\gamma \cdot d}
  \label{eqn:05}
\end{align}
\noindent berechnet sich die Gradientenfeldstärke unter Beachtung des
gyromagnetischen Verhältnisses eines Protons $\gamma \approx \num{268e6} \:
\frac{\text{rad}}{\si{\second\tesla}}$ zu
$G = \num{0.081} \frac{\si{\tesla}}{\si{\meter}}$. \\
\noindent Um daraus die Diffusionskonstante $D$ berechnen zu können, ist die
Relaxationszeit $T_D$ zu berechnen. Die Messungen der über mehrere
Größenordnungen variierenden Pulsabstände lassen sich durch eine Funktion
\begin{align}
  M(\tau) = M_0 \cdot \exp{\left(- \frac{2 \cdot \tau}{T_2}\right)} \cdot \exp{\left(- \frac{2 \cdot \tau^3}{a}\right)} + M_1
 \label{eqn:06}
\end{align}
\noindent approximieren. In Abbildung \ref{fig:03} ist dieser Zusammenhang
graphisch veranschaulicht. Dabei ergeben sich die Parameter
\begin{align*}
  M_0 &=  (0.890  \pm  0.005)          \si{\volt} \\
  M_1 &=  (0.0390  \pm  0.005)         \si{\volt} \\
  a &=  (3.175e-06  \pm  5.792e-08) \si{\second^3}.
\end{align*}
\FloatBarrier
\begin{figure}
  \includegraphics{build/tau_D.pdf}
  \caption{Graphische Darstellung der aus den .csv Dateien extrahierten
  Amplituden in Abhängigkeit vom Pulsabstand $\tau$}
  \label{fig:03}
\end{figure}
\FloatBarrier
\noindent Die verwendeten Messwerte lassen sich Tabelle \ref{table:02} entnehmen.
\FloatBarrier
\begin{table}
\centering
\caption{Gemessene Zählraten für verschiedene Verzögerungszeiten $t_\text{Verz.}$ bei einer fest gewählten
Referenzzeit $t_\text{Ref.} = \SI{20}{\nano\second}$.}
\sisetup{table-format=3.0}
\begin{tabular}{c c c | c c c}
\toprule
\multicolumn{1}{c}{$t_\text{Verz.} \:/\: \si{\nano\second}$} & \multicolumn{1}{c}{$n_\text{links}$}
& \multicolumn{1}{c}{$n_\text{rechts}$} & \multicolumn{1}{c}{$t_\text{Verz.} \:/\: \si{\nano\second}$}
& \multicolumn{1}{c}{$n_\text{links}$} & \multicolumn{1}{c}{$n_\text{rechts}$}  \\
\midrule
0   & 184 & 184 & 10 & 206 & 169 \\
1   & 228 & 186 & 12 & 209 & 196 \\
1.5 & 230 & 208 & 16 & 127 & 195 \\
2   & 200 & 202 & 18 & 112 & 128 \\
4   & 252 & 214 & 20 & 56  & 104 \\
6   & 239 & 188 & 22 & 39  & 70  \\
8   & 217 & 203 & 24 & 9   & 27  \\
\bottomrule
\end{tabular}
\label{tabular_02}
\end{table}

\FloatBarrier
\noindent Die Relaxationszeit $T_D$ lässt sich dann berechnen über
$T_D = \frac{a}{\tau^3}$, um daraus schließlich Rückschlüsse auf die
Diffusionskonstante zu generieren. Es gilt
\begin{align*}
  D = (2.006e-09 \pm 3.660e-11) \frac{\si{\meter^2}}{\si{\second}}.
\end{align*}
\noindent Aus der Einstein-Stokes-Gleichung \ref{eqn:02} kann daraus unter Verwendung der
Parameter $T = \SI{293.75}{\kelvin}$ (gemessen) und $\eta = \SI{10e-3}{\pascal\second}$
\cite{Viskosität_Wasser} der Molekülradius zu
\begin{align*}
 r = (1.073\pm 1.957e{-2}) \text{\r{A}}
\end{align*}
\noindent bestimmen. Als Vergleichswert lässt sich aus der Gleichung
\begin{align}
  r = \sqrt[3]{\frac{3 \cdot M}{\rho \cdot N_A \cdot 4 \cdot \pi}}
  \label{eqn:07}
\end{align}
\noindent nach \cite{radius} bestimmen. Für $M = \SI{18.015}{\gram\per\mol}$
und $\rho = 1 \: \frac{\si{\gram}}{\si{\cubic\centi\meter}}$
\cite{Viskosität_Wasser} und die Avogadro-Konstante $N_A$ bedeutet dies einen
Radius
\begin{align*}
  r_\text{Theorie} = \num{1.742} \text{\r{A}}.
\end{align*}
