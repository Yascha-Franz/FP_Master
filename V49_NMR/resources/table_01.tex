\begin{table}
\centering
\caption{Messwerte der $T_1$ Relaxationszeit.Die letzten beide Werte (bei 8000 und 9000) sind nicht im Fit miteinbezogen.}
\sisetup{table-format=2.5}
\begin{tabular}{S[table-format=3.0]  S[table-format=4.0] | S[table-format=3.0]  S[table-format=4.0]  }
\toprule
\multicolumn{1}{c}{$\tau \:/\: \si{\milli\second}$} & \multicolumn{1}{c}{$U\:/\: \si{\milli\volt}$}
& \multicolumn{1}{c}{$\tau \:/\: \si{\milli\second}$} & \multicolumn{1}{c}{$U\:/\: \si{\milli\volt}$}  \\
\midrule
1   & -981 & 500  & -625 \\
2   & -981 & 1000 & -356 \\
4   & -981 & 1500 & -137 \\
8   & -981 & 2000 & 152  \\
15  & -981 & 3000 & 402  \\
30  & -981 & 5000 & 742  \\
70  & -906 & 8000 & 177  \\
100 & -895 & 9000 & 167  \\
200 & -806 \\
\bottomrule
\end{tabular}
\label{table:01}
\end{table}
