\noindent Die Messergebnisse zeigen eine gute Übereinstimmung mit der in der
Theorie bereits erwähnten Lebensdauer der kosmischen Myonen. Liegt diese nach
\cite{grupen} bei einem Wert von $\tau_\text{Lit.} = \SI{2.2}{\micro\second}$,
so liegt das Messergebnis $\tau_\text{Messung} = \SI{2.07}{\micro\second}$ in
einem für die Versuchsdurchführung vertretbaren Unsicherheitsbereich. \\
Die Messunsicherheiten können dabei die Annahme eines poissonverteilten
Untergrundes hinsichtlich der Berechnung des Untergrundes sein, auf der
experimentellen Sicht eine nicht optimale Einstellung der Verzögerungszeiten
oder statistische Messunsicherheiten. 
