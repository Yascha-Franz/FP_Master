\noindent Im Wesentlichen besteht der Versuchsaufbau aus einem Szintillatortank
mit einem Volumen $V = \SI{50}{\liter}$ und zwei angeflanschten PMTs, sowie
einem Rack mit verschiedenen elektrischen Bauteilen zur Signalverarbeitung. Zu
diesen Geräten gehören Verzögerungsleitungen (verschieden lange Kabelverbindungen
mit Verzögerunsgleistungen im Bereich einiger Nanosekunden), Diskriminatoren
zur Festlegung von Schwellspannungen, einem Doppelimpulsgenerator, einer
Koinzidenzschaltung, Impulszählgeräten, einem Monoflop, einem
Time-Amplitude-Converter (TAC) sowie einem Multi-Channel-Analyzer (MCA), dessen
Signale von einem PC ausgewertet werden. Zusätzlich steht ein Oszilloskop bereit.
Die Bauelemente sind alle in Abbildung \ref{fig:02} aufgeführt.  \\
\newline
\noindent Die Ausgängssignale der Photomultiplier sind über
Verzögerungsleitungen mit Diskriminatoren verbunden und werden anschließend an
eine Koinzidenzschaltung weitergeleitet. Diese ist ebenfalls mit einem
Doppelimpulsgenerator verbunden. Die Koinzidenzschaltung leitet lediglich im
Falle eines simultanen Signaleingangs der Signale zweier PMT ein Signal weiter,
zu Kalibrierungszwecken kann der Doppelimpulsgenerator das erforderliche zweite
Eingangssignal geben. \\
\noindent Am Anschluss an die Koinzidenzschaltung folgen zwei AND-Gitter, wovon
eines direkt, ein zweites erst nach einer weiteren Verzögerungsleitung und einem
Monoflip vom Signal erreicht wird. Der Monoflip dient dabei einer Eingrenzung
der Suchzeit, innerhalb welcher die Myonzerfälle zu erwarten sind. Die
Verzögerungsleitung soll eine Verzögerungszeit $\Delta t = \SI{30}{\nano\second}$
gewährleisten. Die beiden AND-Gitter, die ein Signal jeweils aus der
Koinzidenzschaltung direkt beziehen, erhalten ihr zweites Signal aus dem
Monoflop. Dieser erhält sein Signal um die Zeitspanne $\Delta t$ zeitversetzt.
Wird er getriggert, sperrt er für die Suchzeit $T_s$ das Startsignal am ersten
AND-Gitter und gibt es nach der Suchzeit wieder frei. Ein Signaleingang
innerhalb der Suchzeit hingegen schaltet ein Stopp-Signal. Die Start und
Stop-Signale werden jeweils einzeln gezählt, die Signalverarbeitung geschieht
über den TAC und den MCA am PC \cite{anleitung}.   
