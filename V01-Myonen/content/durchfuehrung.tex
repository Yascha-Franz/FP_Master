\noindent Die Durchführung des Versuchs zur Lebensdauer kosmischer Myonen
beschränkt sich im Wesentlichen auf den aufbau der elektrischen Schaltung nach
Abbildung \ref{fig:02} und auf das Einstellen der Verzögerungszeiten sowie der
Schwellspannungen der Diskriminatoren. \\
\noindent Zu Beginn der Justage sind die PMT mit Hochspannung zu versorgen.
Dabei ist zu beachten, das die Amplituden der Spannungsimpulse der beiden
PMT verschieden groß sind, was mit einem Oszilloskop überprüft werden kann. \\
\noindent Anschließend sind die Schwellspannungen an den Diskriminatoren mithilfe
eines Schrubendrehers so einzustellen, dass an beiden Diskriminatorausgängen etwa
$n_\text{Diskr.} = 30$ Impulse pro Sekunde gezählt werden. Danach wird die
Pulsdauer $\Delta t = \SI{10}{\nano\second}$ eingestellt und mit dem Oszilloskop
überprüft. Durch Variation der Verzögerungsleitungen wird im nächsten Schritt die
Koinzidenzapparatur justiert und die Impulsrate beobachtet. Der Messbereich sollte
dafür so gewählt werden, dass die Halbwertsbreite der Verteilung bestimmt werden
kann. Während der gesamten Versuchsdurchführung ist eine Verzögerung der
Signaleingänge zu wählen, deren Ereignisrate im Bereich von $20$ Counts pro
Sekunde liegt. \\
\noindent Im Anschluss daran ist die vollständige Schaltung nach Abbildung
\ref{fig:02} herzustellen. An die Suchzeit $T_S$ am Monoflop angepasst ist der
Messbereich des TAC zu wählen. Die abschließende Kalibrierung des MCA erfolgt
mithilfe des Doppelimpulsgenerators. Über zehn Messwerte im Abstandsbereich
zwischen den Pulsen im Intervall $t_\text{Puls} \in
[\SI{0.3}{\micro\second},\SI{9.9}{\micro\second}]$ werden die Zählraten in den
einzelnen Messkanälen bestimmt. \\
\noindent Die eigentliche Messung startet nach einem Reset der während der
Kalibration registrierten Zählraten und wird über mehrere Tage durchgeführt
\cite{anleitung}. 
