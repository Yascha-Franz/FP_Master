\noindent Für die Fehlerrechung wird die empirische Standardabweichung
\begin{equation}
  \sigma = \sqrt{\frac{1}{n-1} \cdot \sum_{i=1}^n(x_i-\overline{x})^2}
  \label{eqn:Stdabweichung}
\end{equation}
und die Gaußsche Fehlerfortpflanzung
\begin{equation}
  u_y = \sqrt{\sum_{i=1}^n\left(\frac{\delta y}{\delta x_i}u_x\right)^2}
  \label{eqn:gauß}
\end{equation}
verwendet. Zur Auswertung werden verschiedene Programme wie matplotlib
\cite{matplotlib}, NumPy \cite{numpy}, SciPy \cite{scipy} und Uncertainties
\cite{uncertainties} hinzugezogen. \\
\newline
\noindent Um zu gewährleisten, dass die Ereignisse simultan verarbeitet werden
können, ist es erforderlich, die Verzögerungen der PMT-Signale so zu wählen, dass
sie simultan an der Koinzidenzapparatur verarbeitet werden können. Hierzu werden
die für zwei Justagedurchgänge durchgeführt, in denen je eine Verzögerungszeit
festgehalten wird. Die Messwerte sind für je zwei verschiedene Flanken in den
Tabellen \ref{tabular_01} und Tabelle \ref{tabular_02} aufgeführt. \\
\FloatBarrier
\begin{table}
\centering
\caption{Gemessene Zählraten für verschiedene Verzögerungszeiten $t_\text{Verz.}$ bei einer fest gewählten
Referenzzeit $t_\text{Ref.} = \SI{10}{\nano\second}$.}
\sisetup{table-format=3.0}
\begin{tabular}{c c c | c c c}
\toprule
\multicolumn{1}{c}{$t_\text{Verz.} \:/\: \si{\nano\second}$} & \multicolumn{1}{c}{$n_\text{links}$}
& \multicolumn{1}{c}{$n_\text{rechts}$} & \multicolumn{1}{c}{$t_\text{Verz.} \:/\: \si{\nano\second}$}
& \multicolumn{1}{c}{$n_\text{links}$} & \multicolumn{1}{c}{$n_\text{rechts}$}  \\
\midrule
0   & 238 & 238 & 6   & 173 & 168 \\
1   & 239 & 218 & 8   & 116 & 125 \\
1.5 & 243 & 243 & 10  & 89  & 89  \\
2   & 219 & 230 & 12  & 43  & 41  \\
4   & 215 & 200 \\
\bottomrule
\end{tabular}
\label{tabular_01}
\end{table}

\FloatBarrier
\FloatBarrier
\begin{table}
\centering
\caption{Gemessene Zählraten für verschiedene Verzögerungszeiten $t_\text{Verz.}$ bei einer fest gewählten
Referenzzeit $t_\text{Ref.} = \SI{20}{\nano\second}$.}
\sisetup{table-format=3.0}
\begin{tabular}{c c c | c c c}
\toprule
\multicolumn{1}{c}{$t_\text{Verz.} \:/\: \si{\nano\second}$} & \multicolumn{1}{c}{$n_\text{links}$}
& \multicolumn{1}{c}{$n_\text{rechts}$} & \multicolumn{1}{c}{$t_\text{Verz.} \:/\: \si{\nano\second}$}
& \multicolumn{1}{c}{$n_\text{links}$} & \multicolumn{1}{c}{$n_\text{rechts}$}  \\
\midrule
0   & 184 & 184 & 10 & 206 & 169 \\
1   & 228 & 186 & 12 & 209 & 196 \\
1.5 & 230 & 208 & 16 & 127 & 195 \\
2   & 200 & 202 & 18 & 112 & 128 \\
4   & 252 & 214 & 20 & 56  & 104 \\
6   & 239 & 188 & 22 & 39  & 70  \\
8   & 217 & 203 & 24 & 9   & 27  \\
\bottomrule
\end{tabular}
\label{tabular_02}
\end{table}

\FloatBarrier
\noindent Die gemessenen Werte lassen sich dabei flankenweise betrachten. Das
bedeutet, dass die Messwerte der linken Flanke negativen Zeitverzögerungswerten
zugeordnet werden, die auf der rechten Seite dementsprechend den positiven
Zeitverzögerungswerten. Zusätzlich zeigen beide Messungen eine Plateaubildung.
Graphische Veranschaulichungen der Messwerte sind in den Abbildungen \ref{fig:04}
und Abbildung \ref{fig:05} dargestellt.
\FloatBarrier
\begin{figure}
  \centering
  \includegraphics{bulid/koinzidenz_10ns.pdf}
  \caption{Messwerte der Zeitverzögerung bei einer Referenzverzögerungszeit $\SI{10}{\nano\second}$.}
  \label{fig:04}
\end{figure}
\FloatBarrier
\FloatBarrier
\begin{figure}
  \centering
  \includegraphics{bulid/koinzidenz_20ns.pdf}
  \caption{Messwerte der Zeitverzögerung bei einer Referenzverzögerungszeit $\SI{20}{\nano\second}$.}
  \label{fig:05}
\end{figure}
\FloatBarrier
\noindent Die Ausgleichsgeraden für die Flanken und die jeweiligen Plateaus
besitzen dabei die Form
\begin{align}
  f(x) = m \cdot x + b,
  \label{eqn:08}
\end{align}
\noindent die Parameter für die Referenzverzögerungszeit $\SI{10}{\nano\second}$
(siehe Abbildung \ref{fig:05}) lauten
\begin{align*}
  m_\text{links}   &= \num{(20.85 \pm 1.93)} \si{\nano\second^{-1}} \\
  b_\text{links}   &= \num{(292.90 \pm 17.98)} \\
  m_\text{Plateau} &= \num{0} \\
  b_\text{Plateau} &= \num{(230.63 \pm 11.01)} \\
  m_\text{rechts}  &= \num{(-19.85 \pm 0.73)} \si{\nano\second^{-1}} \\
  b_\text{rechts}  &= \num{(283.40 \pm 6.21)}. \\
\end{align*}
\noindent Analog lassen sich mit Gleichung \ref{fig:08} die Parameter
für die Referenzverzögerungszeit $\SI{10}{\nano\second}$
(siehe Abbildung \ref{fig:06}) bestimmen zu
\begin{align*}
  m_\text{links}   &= \num{(15.45 \pm 1.67)} \si{\nano\second^{-1}} \\
  b_\text{links}   &= \num{(377.60 \pm 33.65)} \\
  m_\text{Plateau} &= \num{0} \\
  b_\text{Plateau} &= \num{(207.0 \pm 20.14)} \\
  m_\text{rechts}  &= \num{(-16.85 \pm 1.50)} \si{\nano\second^{-1}} \\
  b_\text{rechts}  &= \num{(436.10 \pm 31.73)}. \\
\end{align*}
\noindent Die genäherten Plateaus entsprechen dabei den Verzögerungszeiten, für
welche eine maximale Zählrate gemessen werden kann. Daher wird eine
Verzögerungszeit gewählt, für welche 
