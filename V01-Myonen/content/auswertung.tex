\noindent Für die Fehlerrechung wird die empirische Standardabweichung
\begin{equation}
  \sigma = \sqrt{\frac{1}{n-1} \cdot \sum_{i=1}^n(x_i-\overline{x})^2}
  \label{eqn:Stdabweichung}
\end{equation}
und die Gaußsche Fehlerfortpflanzung
\begin{equation}
  u_y = \sqrt{\sum_{i=1}^n\left(\frac{\delta y}{\delta x_i}u_x\right)^2}
  \label{eqn:gauß}
\end{equation}
verwendet. Zur Auswertung werden verschiedene Programme wie matplotlib
\cite{matplotlib}, NumPy \cite{numpy}, SciPy \cite{scipy} und Uncertainties
\cite{uncertainties} hinzugezogen. \\
\newline
\noindent Um zu gewährleisten, dass die Ereignisse simultan verarbeitet werden
können, ist es erforderlich, die Verzögerungen der PMT-Signale so zu wählen, dass
sie simultan an der Koinzidenzapparatur verarbeitet werden können. Hierzu werden
die für zwei Justagedurchgänge durchgeführt, in denen je eine Verzögerungszeit
festgehalten wird. Die Messwerte sind für je zwei verschiedene Flanken in den
Tabellen \ref{tabular_01} und Tabelle \ref{tabular_02} aufgeführt. \\
\FloatBarrier
\begin{table}
  \centering
  \caption{Polarisationsabhängigkeit der Intensitätsextrema.}
  \label{tabular_01}
  \begin{tabular}{c c c c | c c c c}
    \toprule
   \multicolumn{1}{c}{$\Theta$} & \multicolumn{1}{c}{$U_\text{min}$} & \multicolumn{1}{c}{$U_\text{max}$}
    & \multicolumn{1}{c}{$K$} & \multicolumn{1}{c}{$\Theta$} & \multicolumn{1}{c}{$U_\text{min}$}
    & \multicolumn{1}{c}{$U_\text{max}$} & \multicolumn{1}{c}{$K$}\\
   \midrule
   \SI{-15}{\degree} & \SI{694}{\milli\volt} & \SI{1710}{\milli\volt} &  {0.42} & \SI{105}{\degree} & \SI{640}{\milli\volt} & \SI{1572}{\milli\volt} &  {0.42} \\
   \SI{0  }{\degree} & \SI{765}{\milli\volt} & \SI{906 }{\milli\volt} &  {0.08} & \SI{120}{\degree} & \SI{306}{\milli\volt} & \SI{2250}{\milli\volt} &  {0.76} \\
   \SI{15 }{\degree} & \SI{377}{\milli\volt} & \SI{712 }{\milli\volt} &  {0.31} & \SI{130}{\degree} & \SI{200}{\milli\volt} & \SI{2670}{\milli\volt} &  {0.86} \\
   \SI{30 }{\degree} & \SI{148}{\milli\volt} & \SI{620 }{\milli\volt} &  {0.61} & \SI{135}{\degree} & \SI{168}{\milli\volt} & \SI{2540}{\milli\volt} &  {0.88} \\
   \SI{40 }{\degree} & \SI{67 }{\milli\volt} & \SI{696 }{\milli\volt} &  {0.82} & \SI{140}{\degree} & \SI{219}{\milli\volt} & \SI{2640}{\milli\volt} &  {0.85} \\
   \SI{45 }{\degree} & \SI{63 }{\milli\volt} & \SI{740 }{\milli\volt} &  {0.84} & \SI{150}{\degree} & \SI{355}{\milli\volt} & \SI{2270}{\milli\volt} &  {0.73} \\
   \SI{50 }{\degree} & \SI{58 }{\milli\volt} & \SI{770 }{\milli\volt} &  {0.86} & \SI{165}{\degree} & \SI{571}{\milli\volt} & \SI{1514}{\milli\volt} &  {0.45} \\
   \SI{60 }{\degree} & \SI{91 }{\milli\volt} & \SI{951 }{\milli\volt} &  {0.83} & \SI{180}{\degree} & \SI{721}{\milli\volt} & \SI{862 }{\milli\volt} &  {0.09} \\
   \SI{75 }{\degree} & \SI{339}{\milli\volt} & \SI{1167}{\milli\volt} &  {0.55} & \SI{195}{\degree} & \SI{317}{\milli\volt} & \SI{704 }{\milli\volt} &  {0.38} \\
   \SI{90 }{\degree} & \SI{809}{\milli\volt} & \SI{1074}{\milli\volt} &  {0.14} \\
\bottomrule
  \end{tabular}
\end{table}

\FloatBarrier
\FloatBarrier
\begin{table}
  \centering
  \caption{Anzahl der Intensitätsextrema je Drehung der Doppelglashalter um $\alpha = \SI{10}{\degree}$.}
  \label{tabular_02}
  \begin{tabular}{c c | c c}
    \toprule
   \multicolumn{1}{c}{$t_\text{Durchgang}$} & \multicolumn{1}{c}{$n_\text{counts}$} & \multicolumn{1}{c}{$t_\text{Durchgang}$} & \multicolumn{1}{c}{$n_\text{counts}$}\\
   \midrule
    \num{1} & \num{37} & \num{6 } & \num{37} \\
    \num{2} & \num{39} & \num{7 } & \num{35} \\
    \num{3} & \num{36} & \num{8 } & \num{37} \\
    \num{4} & \num{34} & \num{9 } & \num{35} \\
    \num{5} & \num{36} & \num{10} & \num{37} \\
\bottomrule
  \end{tabular}
\end{table}

\FloatBarrier
\noindent Die gemessenen Werte lassen sich dabei flankenweise betrachten. Das
bedeutet, dass die Messwerte der linken Flanke negativen Zeitverzögerungswerten
zugeordnet werden, die auf der rechten Seite dementsprechend den positiven
Zeitverzögerungswerten. Zusätzlich zeigen beide Messungen eine Plateaubildung.
Graphische Veranschaulichungen der Messwerte sind in den Abbildungen \ref{fig:04}
und Abbildung \ref{fig:05} dargestellt.
\FloatBarrier
\begin{figure}
  \centering
  \includegraphics{build/koinzidenz_10ns.pdf}
  \caption{Messwerte der Zeitverzögerung bei einer Referenzverzögerungszeit $\SI{10}{\nano\second}$.}
  \label{fig:04}
\end{figure}
\FloatBarrier
\FloatBarrier
\begin{figure}
  \centering
  \includegraphics{build/koinzidenz_20ns.pdf}
  \caption{Messwerte der Zeitverzögerung bei einer Referenzverzögerungszeit $\SI{20}{\nano\second}$.}
  \label{fig:05}
\end{figure}
\FloatBarrier
\noindent Die Ausgleichsgeraden für die Flanken
besitzen dabei die Form
\begin{align}
  f(x) = m \cdot x + b
  \label{eqn:08}
\end{align}
\noindent  und die jeweiligen Plateaus werden über den empirischen Mittelwert bestimmt.
\noindent Die Parameter für die Referenzverzögerungszeit $\SI{10}{\nano\second}$
(siehe Abbildung \ref{fig:04}) lauten
\begin{align*}
  m_\text{links}   &= {(20.85 \pm 1.93)} \si{\nano\second^{-1}} \\
  b_\text{links}   &= {(292.90 \pm 17.98)} \\\\
  b_\text{Plateau} &= {(230.63 \pm 11.01)} \\
  m_\text{rechts}  &= {(-19.85 \pm 0.73)} \si{\nano\second^{-1}} \\
  b_\text{rechts}  &= {(283.40 \pm 6.21)}. \\
\end{align*}
\noindent Analog lassen sich mit Gleichung \ref{eqn:08} die Parameter
für die Referenzverzögerungszeit $\SI{10}{\nano\second}$
(siehe Abbildung \ref{fig:05}) bestimmen zu
\begin{align*}
  m_\text{links}   &= {(15.45 \pm 1.67)} \si{\nano\second^{-1}} \\
  b_\text{links}   &= {(377.60 \pm 33.65)} \\\\
  b_\text{Plateau} &= {(207.0 \pm 20.14)} \\
  m_\text{rechts}  &= {(-16.85 \pm 1.50)} \si{\nano\second^{-1}} \\
  b_\text{rechts}  &= {(436.10 \pm 31.73)}. \\
\end{align*}
\noindent Die genäherten Plateaus entsprechen dabei den Verzögerungszeiten, für
welche eine maximale Zählrate gemessen werden kann. Daher wird eine
Verzögerungszeit gewählt, für welche einer der Plateauwerte erreicht wird. \\
\newline
\noindent Die Zuordnung der Impulse zu den jeweiligen Channels, in denen sie
registriert werden, erfolgt auf Basis der in Tabelle \ref{tabular_03} dargestellten
Messwerte. \\
\FloatBarrier
\begin{table}
\centering
\caption{Kalibrationsmesswerte des MCA.}
\sisetup{table-format=3.0}
\begin{tabular}{c c | c c}
\toprule
\multicolumn{1}{c}{$t_\text{Kal.} \:/\: \si{\micro\second}$} & \multicolumn{1}{c}{$n$}
& \multicolumn{1}{c}{$t_\text{Kal.} \:/\: \si{\micro\second}$} & \multicolumn{1}{c}{$n$}  \\
\midrule
0.9 & 35  & 5.9 & 265 \\
1.9 & 81  & 6.9 & 311 \\
2.9 & 127 & 7.9 & 357 \\
3.9 & 173 & 8.9 & 404 \\
4.9 & 219 & 9.9 & 450 \\
\bottomrule
\end{tabular}
\label{tabular_03}
\end{table}

\FloatBarrier
\noindent Aus den in Tabelle \ref{tabular_03} angegebenen Frequenzwerten und den
zugehörigen Zählraten lässt sich die in Abbildung \ref{fig:06} dargestellte
lineare Ausgleichsfunktion darstellen. Diese ist ebenfalls vom in Gleichung
\ref{eqn:08} dargestellten Funktionstyp und besitzt die Ausgleichsparameter
\begin{align*}
  m_\text{Kal.} &= (0.02 \pm 1.6 \cdot 10{-5}) \frac{\si{\micro\second}}{\text{Channel}}  \qquad \text{und}\\
  b_\text{Kal.} &= (0.15 \pm 0.004) \si{\micro\second}.
\end{align*}
\FloatBarrier
\begin{figure}
  \centering
  \includegraphics{build/Zeit_Channel_Kalibration.pdf}
  \caption{Pulsdauern in Abhängigkeit vom Registrationschannel.}
  \label{fig:06}
\end{figure}
\FloatBarrier
\noindent Um die Lebensdauer der kosmischen Myonen zu bestimmen, ist es
erforderlich, von der Gesamtzählrate der Myonen den Untergrund zu subtrahieren.
Der Untergrund entspricht dabei einer Poissonverteilung
\begin{equation}
  N(t) = N_0 \cdot P_{\text{Poisson}}(t)
\end{equation}
mit einer
Zerfallskonstanten $\lambda = r \cdot t$ bestehend aus dem Produkt der gesamten gemessenen Rate an Startereignissen $r = 9530546 \text{counts}/266036 \si{\second}$ und der
dem Channel zugeordneten Zeit $t$ und lässt sich als Wahrscheinlichkeit interpretieren, zwei Myonen
innerhalb des für einen Zerfall zu erwartenden Zeitraumes zu detektieren. Der Untergrund pro Channel ist in Abbildung \ref{fig:07} dargestellt.  \\
\FloatBarrier
\begin{figure}
  \centering
  \includegraphics{build/Myonenuntergrund.pdf}
  \caption{Aus der Poissonverteilung für eine Anzahl $n = 9530546$ ermittelte
  Untergrundrate detektierter Myonen}
  \label{fig:07}
\end{figure}
\FloatBarrier
\noindent Mithilfe des Untergrundes lässt sich anschließend die Lebensdauer der
kosmischen Myonen bestimmen. Das Gesamtspektrum ist mit bilanzierten Untergrund
in Abbildung \ref{fig:08} dargestellt. \\
\FloatBarrier
\begin{figure}
  \centering
  \includegraphics{build/Gesamtspektrum.pdf}
  \caption{Exponentielle Abnahme der Zählraten kosmischer Myonen in Abhängigkeit
  von der Zerfallszeit.}
  \label{fig:08}
\end{figure}
\FloatBarrier
\noindent Die Ausgleichskurve in Abbildung \ref{fig:08} zeigt einen
exponentiellen Verlauf der Form
\begin{align}
  N(t) = a \cdot \exp{\frac{t}{\tau}} + c.
  \label{eqn:09}
\end{align}
\noindent Die Ausgleichsrechnung des Zerfallsspektrums liefert die Parameter
\begin{align*}
  a &= 142.03 \pm 1.43 \qquad \text{und} \\
  c &= -5.42 \pm 0.55,
\end{align*}
\noindent sowie eine Lebensdauer der kosmischen Myonen von
\begin{align*}
  \tau = (2.07 \pm 0.04) \si{\micro\second}.
\end{align*}
