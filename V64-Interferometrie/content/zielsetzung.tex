Materialspezifische Eigenschaften wie beispielsweise Brechungsindizes lassen
sich mithilfe von Interferenzeffekten mit sehr hoher Präzision untersuchen.
Mithilfe eines Sagnac-Interferometers sollen der Brechungsindex von Glas und
der Brechungsindex von Luft unter verschiedenen Drücken untersucht werden.
Ein Schwerpunkt bei der Durchführung dieses Experimentes soll dabei auf auf der
korrekten Justage des optischen Strahlengangs des Interferometers liegen.
