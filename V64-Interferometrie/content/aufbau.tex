\noindent Der prinzipielle Aufbau des Sagnac-Interferometers ist in Abbildung
\ref{fig:01} dargestellt. Ein Helim-Neon-Laser mit einer Wellenlänge
$\lambda = \SI{632.990}{\nano\meter}$ emittiert kohärente Lichtwellen. Diese
werden von einem um $\SI{45}{\degree}$ gedrehten Spiegel (ortsfest und in seiner
Neigung einstellbar) auf einen weiteren, um $\SI{45}{\degree}$ gegen den
Strahlengang gedregten Spiegel transmittiert. Dieser Spiegel lässt sich sowohl
zur Justage neigen, als auch in Strahlrichtung axial verschieben. \\
\noindent Im Anschluss daran befindet sich im Strahlengang ein Strahlteilerwürfel
PBSC (\textit{polarizing beam splitter cube}). Dieser erzeugt aus dem linear
polarisierten Licht zwei orthogonal zueinander polarisierte Lichtstrahlen, die
in einem Winkel von $\SI{90}{\degree}$ gegeneinander verschoben entweder
transmittiert oder Reflektiert werden. Durch die Orientierung des PBSC im
Strahlengang von $\SI{45}{\degree}$ bezogen auf den Strahlengang, passiert ein
Teilstrahl den PBSC geradlinig, der zweite Teilstrahl wird um $\SI{90}{\degree}$
gegenüber dem anderen Strahl abgelenkt. Drei weitere, ortsfeste, justierbare
Spiegel bilden zusammen mit dem PBSC einen quadratischen Strahlengang. \\
\noindent Ein Polarisationsfilter zwischen dem axial verschiebbaren Spiegel und
dem PBSC legt die Polarisationsrichtung eindeutig fest,ein zweiter
Polarisationsfilter hinter dem eigentlichen Interferometer trennt die
orthogonal polarisierten Strahlen und ermöglicht die Beobachtung von
Interferenzmustern. \\
\noindent Je nach Experiment lassen sich verschiedene optische Medien in den
Strahlengang einbringen. Im durchgeführten Experiment sind dies zwei um
$\SI{10}{\degree}$ zueinander geneigt stehende Glasplättchen der Dicke
$T=\SI{1}{\milli\meter}$, sowie eine Gasdruckröhre der Länge
$L = \SI{100}{\milli\meter}$, welche mit einer Vakuumpumpe verbunden ist, über
die der Umgebungsdruck auf wenige Millibar reduziert und anschließend durch
Belüftung der Apparatur unter Beobachtung mit einem Manometer an das Normalniveau
angeglichen werden kann. \\
\noindent Um eine reproduzierbare Versuchsdurchführung zu gewährleisten, kann der
Versuchsaufbau durch ein Gehäuse aus Plexiglas abgedeckt werden, um eventuell
auftretende Druckschwankungen der Umgebungsluft im Strahlengang der Teilstrahlen
zu reduzieren. Durch ein Thermometer wird die Temperatur bei der
Versuchsdurchführung überwacht.  \\
\noindent Detektiert werden die Interferenzerscheinungen durch zwei Dioden
sowie einen Modern Interferometry Controller. Dieser misst die Differenz beider
Signale und gibt als Ausgabe die Anzahl $M$ der Interferenzmaxima aus. Ein
Multimeter kann zum Auslesen der Spannungen in den Interferenzextrema genutzt
werden. \\
