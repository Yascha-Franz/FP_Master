Abschließend sollen die berechneten Brechungsindizes mit aktuellen
Literaturwerten verglichen werden, um die Abweichung zu quantisieren. Mögliche
Literaturwerte sind in Tabelle \ref{tabular_05} aufgelistet, wobei die genaue
Zusammensetzung des verwendeten Glases nicht bekannt ist, ebensowenig wie der
Einfluss der Gasdruckröhre selbst (Einlass- und Auslassfenster). \\
\FloatBarrier
\begin{table}
  \centering
  \caption{Aktuelle Litraturwerte für die Brechungsindizes von Glas und Luft.}
  \label{tabular_05}
  \begin{tabular}{c c c c c}
    \toprule
   \multicolumn{1}{c}{Medium} & \multicolumn{1}{c}{$n$} & \multicolumn{1}{c}{Temperatur $T$}
   & \multicolumn{1}{c}{Druck $p$} & \multicolumn{1}{c}{Quellennachweis}\\
   \midrule
    Glas & \num{1.458}  & \SI{20}{\celsius} & \num{1033} & \text{\cite{hecht}} \\
    Luft & \num{1.000293} & \SI{0}{\celsius} & \num{1033} & \text{\cite{hecht}} \\
\bottomrule
  \end{tabular}
\end{table}

\FloatBarrier
\noindent Die berechneten Brechungsindizes wurden Auf Grundlage von Messungen bei
einer gemessenen Temperatur von $T = \SI{20.5}{\celsius}$ gemessen. Für den
berechneten Brechungsindex von Glas $n_\text{Glas} = \num{1.00000128}$ beträgt
die Abweichung etwa $\SI{50}{\percent}$, bei der Messung des Brechungsindexes
von Luft zu beispielsweise $n_\text{Luft} = \num{1.00975}$ beträgt die
Abweichung zum Literaturwert etwa $\SI{1}{\percent}$. Zu beachten ist bei den
Messungen allerdings, dass die angegebenen Literaturwerte bei anderen
Temperaturen bestimmt wurden. Zusätzlich lassen sich Beeinflussungen der
optischen Weglänge durch Druckschwankungen der Luft im Strahlengang des
Interferometers nicht quantifizieren. \\
\noindent Wesentlich für die gemessenen und berechneten Ungenauigkeiten ist
allerdings die Versuchsdurchführung selbst. Wie die Berechnungen des Kontrasts
gezeigt haben, liegt dieser lediglich bei etwa $\SI{88}{\percent}$ des
Idealwertes. Ferner wurde durch die Einlassmethode durch das Öffnen des Ventils
der Druck ungleichmäßig erhöht, was zu Messungenauigkeiten geführt haben kann.
