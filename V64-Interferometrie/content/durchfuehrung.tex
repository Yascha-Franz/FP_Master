\label{durchfuehrung}
\noindent Wesentlicher Bestandteil des Experimentes zur Interferometrie mit dem
Sagnac-Interferometer ist die Justage des Strahlenganges. Um eine gute
Signalqualität zu erreichen, ist es erforderlich, dass Interferenzminima als
vollständige Auslöschung von Licht beobachtet werden. Dazu müssen die im PBSC
geteilten Teilstrahlen exakt ausgerichtet werden. Durch Justageplatten mit
jeweils drei Löchern, die auf die Standfüße der Spiegel gesteckt werden können,
lässt sich die Justage kontrollieren. Diese erfolgt über Stellschrauben an den
Spiegeln, oder über Ausgleichsplättchen, welche unter die Spiegel gelegt werden,
um einen Höhenausgleich zu erreichen. Um Interferenzen erreichen zu können,
missen die Lichtwege sich um ein ganzzahliges Vielfaches einer halben Wellenlänge
unterscheiden. Dazu wird der initiale Lichtstrahl vor dem Interferometer durch
den axial verschiebbaren Spiegel in zwei Strahlen aufgeteilt. Beide Lichtstrahlen
durchlazfen anschließend das Interferometer als gegenläufige Teilstrahlen, wobei
die Teilstrahlen durch in den Strahlengang eingebrachte, lichtbrechende Medien
(Glasplättchen unter verschiedenen Winkeln für beide Strahlen, eine Druckkammer
für nur einen Strahl) eine Änderung ihrer Phase erfahren. Die Qualität der
Interferenz lässt sich durch den Kontrastwert $K$ kontrollieren. Sobald ein
Kontrastwert $K$ nahe $1$ erreicht wird, ist können die Brechungsindizes
untersucht werden. \\
\noindent In Abhängigkeit vom Polarisationswinkel $\Theta$ werden dazu die
Glasplättchen im Strahlengang gedreht. Dadurch werden Interferenzmaxima und
Interferenzminima durchlaufen und die zugehörigen maximal- und Minimalspannungen
Spannungen, registriert von einer Diode, gemessen mit dem Multimeter, notiert.
Aus den gemessenen Werten lässt sich der Kontrast $K$ bestimmen. Der Winkel, für
den der Kontrast $K$ einen Wert nahe der $1$ erreicht wird im Anschluss daran für
die Untersuchung der Brechungsindizes am Polarisator eingestellt. \\
\noindent In Abhängigkeit vom Drehwinkel $\Delta \varphi = \SI{10}{\degree}$ des
Doppelglashalters im Strahlengang wird dazu die Anzahl $M$ der Interferenzmaxima
am Modern Interferometry Controller gezählt.  \\
\noindent Mit Gasdruckröhre in einem der Strahlengänge und ohne Doppelglashalter
wird anschließend der Brechungsindex in Abhängigkeit von Luft unter verschiedenen
Drücken mit zehn Wiederholungen untersucht. Mithilfe der Vakuumpumpe wird der Druck innerhalb der
% Kammer auf etwa $p = \SI{10}{\milli\bar}$ reduziert und das Belüftungsventil der
Vakuumpumpe vorsichtig geöffnet, sodass bei steigendem Druck in der Gasdruckröhre
% in Schritten von $p = \SI{50}{\milli\bar}$ in drei Durchgängen abgelesen, bis
% der Umgebungsdruck $p_\text{Luft} \approx \SI{1000}{\milli\bar}$ erreicht ist.
Zusätzlich wird die Temperatur gemessen.
