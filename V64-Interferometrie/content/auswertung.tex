% Für die Fehlerrechung wird die empirische Standartabweichung
% \begin{equation}
%   \sigma = \sqrt{\frac{1}{n-1} \cdot \sum_{i=1}^n(x_i-\overline{x})^2}
%   \label{eqn:Stdabweichung}
% \end{equation}
% und die Gaußsche Fehlerfortpflanzung
% \begin{equation}
%   u_y = \sqrt{\sum_{i=1}^n\left(\frac{\delta y}{\delta x_i}u_x\right)^2}
%   \label{eqn:gauß}
% \end{equation}
% verwendet.
% \begin{figure}
%   \centering
%   \includegraphics{build/plot.pdf}
%   \caption{Plot}
%   \label{fig:plot}
% \end{figure}
\noindent Die Qualität der Interferenzsignale lässt sich durch den Kontrast $K$
quantifizieren. Um diesen zu ermitteln, wird der initiale Polarisationsfilter
in einem Intervall $\Theta \in [\SI{345}{\degree}, \SI{195}{\degree}]$
in Schritten von je $\theta = \SI{15}{\degree}$ gedreht und die Spannungsmaxima
gemessen. Die Messwerte sind in Tabelle \ref{tabular_01} dargelegt. In den
Bereichen um Winkel, an denen der Kontrast besonders auffällig ist, werden
zusätzliche Messungen durchgeführt. \\
\FloatBarrier
\begin{table}
\centering
\caption{Gemessene Zählraten für verschiedene Verzögerungszeiten $t_\text{Verz.}$ bei einer fest gewählten
Referenzzeit $t_\text{Ref.} = \SI{10}{\nano\second}$.}
\sisetup{table-format=3.0}
\begin{tabular}{c c c | c c c}
\toprule
\multicolumn{1}{c}{$t_\text{Verz.} \:/\: \si{\nano\second}$} & \multicolumn{1}{c}{$n_\text{links}$}
& \multicolumn{1}{c}{$n_\text{rechts}$} & \multicolumn{1}{c}{$t_\text{Verz.} \:/\: \si{\nano\second}$}
& \multicolumn{1}{c}{$n_\text{links}$} & \multicolumn{1}{c}{$n_\text{rechts}$}  \\
\midrule
0   & 238 & 238 & 6   & 173 & 168 \\
1   & 239 & 218 & 8   & 116 & 125 \\
1.5 & 243 & 243 & 10  & 89  & 89  \\
2   & 219 & 230 & 12  & 43  & 41  \\
4   & 215 & 200 \\
\bottomrule
\end{tabular}
\label{tabular_01}
\end{table}

\FloatBarrier
\noindent Die Spannungsextrema lassen sich dabei anstelle der Intensitätsextrema
in Gleichung \ref{eqn:01} einsetzen, sodass als Kontrast für weitere Messungen
der Winkel $\Theta$ gewählt wird, bei welchem für den Kontrast in grober Näherung
$K \approx 1$ gilt. Die Kontrastwerte sind in Abbildung \ref{fig:02} dargestellt.
\FloatBarrier
\begin{figure}
  \centering
  \includegraphics{build/plot_K.pdf}
  \caption{Normierte Kontrastwerte in Abhängigkeit vom Polarisationswinkel $\Theta$.}
  \label{fig:02}
\end{figure}
\FloatBarrier
\noindent Der Verlauf der Ausgleichskurve in Abbiuldung \ref{fig:02} zeigt dabei
deutlich die Form der Sinus-Betragsfunktion \ref{eqn:01b}. \\
\noindent Der zum maximalen Kontrast gehören daher die Werte
\begin{align*}
  K_\text{max.} \approx \num{0.88} \\
  \Theta_0 = \SI{135}{\degree}.
\end{align*}
\subsection{Brechungsindex von Glas}
\noindent Zur Bestimmung des Brechungsindexes von Glas wird in $t = 10$
Durchgängen die Anzahl $n$ der bei der Drehung des Doppelglashalters durchlaufenen
Intensitätsmaxima gezählt. Diese sind in Tabelle \ref{tabular_02} dargelegt.
Der Drehwinkel des Doppelglashalters beträgt dabei in jedem Durchgang $\alpha =
\SI{10}{degree}$. \\
\FloatBarrier
\begin{table}
\centering
\caption{Gemessene Zählraten für verschiedene Verzögerungszeiten $t_\text{Verz.}$ bei einer fest gewählten
Referenzzeit $t_\text{Ref.} = \SI{20}{\nano\second}$.}
\sisetup{table-format=3.0}
\begin{tabular}{c c c | c c c}
\toprule
\multicolumn{1}{c}{$t_\text{Verz.} \:/\: \si{\nano\second}$} & \multicolumn{1}{c}{$n_\text{links}$}
& \multicolumn{1}{c}{$n_\text{rechts}$} & \multicolumn{1}{c}{$t_\text{Verz.} \:/\: \si{\nano\second}$}
& \multicolumn{1}{c}{$n_\text{links}$} & \multicolumn{1}{c}{$n_\text{rechts}$}  \\
\midrule
0   & 184 & 184 & 10 & 206 & 169 \\
1   & 228 & 186 & 12 & 209 & 196 \\
1.5 & 230 & 208 & 16 & 127 & 195 \\
2   & 200 & 202 & 18 & 112 & 128 \\
4   & 252 & 214 & 20 & 56  & 104 \\
6   & 239 & 188 & 22 & 39  & 70  \\
8   & 217 & 203 & 24 & 9   & 27  \\
\bottomrule
\end{tabular}
\label{tabular_02}
\end{table}

\FloatBarrier
\noindent Für die gemessene Anzahl an Intensitätsmaxima während der Drehung
lässt sich mithilfe der Gleichung
\begin{align}
  n_\text{Glas} = \frac{1}{1 - \frac{m \cdot \lambda}{2 \cdot \Theta_0^2}},
  \label{eqn:14}
\end{align}
\noindent wobei $\Theta_0$ dem Winkel $\alpha$ im Bogenmaß entspricht und
$\lambda$ der Wellenlänge des He-Ne-Lasers. Mithilfe der Anzahl der
Intensitätsmaxima $M$ lassen sich über Gleichung \ref{eqn:14} die
Brechungsindizes der einzelnen Messungen bestimmen. Diese sind in Tabelle
\ref{tabular_03} dargestellt. \\
\FloatBarrier
\begin{table}
\centering
\caption{Kalibrationsmesswerte des MCA.}
\sisetup{table-format=3.0}
\begin{tabular}{c c | c c}
\toprule
\multicolumn{1}{c}{$t_\text{Kal.} \:/\: \si{\micro\second}$} & \multicolumn{1}{c}{$n$}
& \multicolumn{1}{c}{$t_\text{Kal.} \:/\: \si{\micro\second}$} & \multicolumn{1}{c}{$n$}  \\
\midrule
0.9 & 35  & 5.9 & 265 \\
1.9 & 81  & 6.9 & 311 \\
2.9 & 127 & 7.9 & 357 \\
3.9 & 173 & 8.9 & 404 \\
4.9 & 219 & 9.9 & 450 \\
\bottomrule
\end{tabular}
\label{tabular_03}
\end{table}

\FloatBarrier
\noindent Gemittelt über die berechneten Brechungsindizes von Glas aus Tabelle
\ref{tabular_03} gilt für den Brechungsindex von Glas
\begin{align}
  n_\text{Glas} = 1 +  1.2843674679352546e-06  \pm  4.760170620035104e-08.
\end{align}
\subsection{Brechungsindex von Gasen}
\noindent Für die Untersuchung des Brechungsindexes von Gasen ergeben sich in
Abhängigkeit vom Druck in der Gasdruckröhre die in Tabelle \ref{tabular_04}
aufgeführten Messwerte. \\
\FloatBarrier
\begin{table}
  \centering
  \caption{Anzahl der Intensitätsmaxima in Abhängigkeit vom Druck $p$.}
  \label{tabular_04}
  \begin{tabular}{c c c c |  c c c c}
    \toprule
   \multicolumn{1}{c}{$p_\text{Luft}$} & \multicolumn{1}{c}{$n_1$} & \multicolumn{1}{c}{$n_2$} & \multicolumn{1}{c}{$n_3$}
   & \multicolumn{1}{c}{$p_\text{Luft}$} & \multicolumn{1}{c}{$n_1$} & \multicolumn{1}{c}{$n_2$} & \multicolumn{1}{c}{$n_3$}\\
   \midrule
    \num{50 } \text{mbar}& \num{0 } & \num{0 } & \num{0 } & \num{550 } \text{mbar} & \num{22} & \num{22} & \num{22} \\
    \num{100} \text{mbar}& \num{3 } & \num{2 } & \num{2 } & \num{600 } \text{mbar} & \num{24} & \num{24} & \num{24} \\
    \num{150} \text{mbar}& \num{5 } & \num{5 } & \num{4 } & \num{650 } \text{mbar} & \num{26} & \num{26} & \num{26} \\
    \num{200} \text{mbar}& \num{7 } & \num{7 } & \num{7 } & \num{700 } \text{mbar} & \num{28} & \num{28} & \num{28} \\
    \num{250} \text{mbar}& \num{9 } & \num{9 } & \num{9 } & \num{750 } \text{mbar} & \num{30} & \num{30} & \num{30} \\
    \num{300} \text{mbar}& \num{11} & \num{11} & \num{11} & \num{800 } \text{mbar} & \num{32} & \num{32} & \num{32} \\
    \num{350} \text{mbar}& \num{13} & \num{13} & \num{13} & \num{850 } \text{mbar} & \num{34} & \num{34} & \num{35} \\
    \num{400} \text{mbar}& \num{15} & \num{15} & \num{15} & \num{900 } \text{mbar} & \num{36} & \num{36} & \num{37} \\
    \num{450} \text{mbar}& \num{17} & \num{17} & \num{17} & \num{950 } \text{mbar} & \num{39} & \num{39} & \num{39} \\
    \num{500} \text{mbar}& \num{19} & \num{19} & \num{19} & \num{1000} \text{mbar} & \num{41} & \num{41} & \num{41} \\
\bottomrule
  \end{tabular}
\end{table}

\FloatBarrier
\noindent Aus den in Tabelle \ref{tabular_04} aufgeführten Messwerten lassen sich
die Brechungsindizes der verschiedenen Messungen mithilfe linearer Ausgleichsrechnung
bestimmen. Gemäß Gleichung \ref{eqn:13} lassen sich die Brechungsindizes für
jede Anzahl an Intensitätsmaxima bestimmen. Abbildung \ref{fig:03} zeigt die
zugehörigen Ausgleichsgeraden. \\
\FloatBarrier
\begin{figure}
  \centering
  \includegraphics{build/plot_n_Gas.pdf}
  \caption{Quadrate der berechneten Brechungsindizes $n$ für verschiedene Drücke
  $p$ innerhalb der Gasröhre}
  \label{fig:03}
\end{figure}
\FloatBarrier
\noindent Die linearen Ausgleichsgeraden folgen dabei der Form
\begin{align}
  f(x) = m \cdot x + b,
\end{align}
wobei sich für die verschiedenen Kurven die Steigungen $m$ und Ordinatenabschnitte
$b$ bestimmen lassen zu
\begin{align*}
  m_1 &= 5.35969743e-07 \pm 9.99978798e-01, \\
  m_2 &= 5.39205999e-07 \pm 9.99976466e-01, \\
  m_3 &= 5.47393386e-07 \pm 9.99972801e-01, \\
  b_1 &= 3.20641925e-09 \pm 1.92050688e-06, \\
  b_2 &= 3.17203888e-09 \pm 1.89991349e-06, \\
  b_3 &= 2.90235414e-09 \pm 1.73835215e-06 \\
\end{align*}
%
%
%
%
\noindent Daraus ergeben sich für die jeweiligen Messungen für den Brechungsindex
von Luft
\begin{align}
  n_1 &= 1.0097618404605855  \pm  1.9041222260301724e-06, \\
  n_2 &= 1.0097623179763553  \pm  1.8837043946012876e-06, \\
  n_3 &= 1.0097646536970344  \pm  1.7235408615137845e-06 \\
\end{align}
