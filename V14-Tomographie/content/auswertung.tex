Die durchgeführten Messungen unterliegen Messunsicherheiten. Für die Bestimmung
dieser Messunsicherheiten werden die empirische Standardabweichung
\begin{equation}
  \sigma = \sqrt{\frac{1}{n-1} \cdot \sum_{i=1}^n(x_i-\overline{x})^2}
  \label{eqn:Stdabweichung}
\end{equation}
und die Gaußsche Fehlerfortpflanzung
\begin{equation}
  u_y = \sqrt{\sum_{i=1}^n\left(\frac{\delta y}{\delta x_i}u_x\right)^2}
  \label{eqn:gauß}
\end{equation}
verwendet. \\
\newline
\noindent Die Wahl der Richtung des Strahlengangs durch den Würfel legt die
Matrix $\textbf{A}$ fest. Für die in Abbildung \ref{fig:04} dargestellten
Richtungen ergibt sich die Matrix
\begin{align}
  \textbf{A} = \begin{pmatrix*}[c]
                 1 & 1 & 1 & 0 & 0 & 0 & 0 & 0 & 0 \\
                 0 & 0 & 0 & 1 & 1 & 1 & 0 & 0 & 0 \\
                 0 & 0 & 0 & 0 & 0 & 0 & 1 & 1 & 1 \\
                 0 & \sqrt{2} & 0 & \sqrt{2} & 0 & 0 & 0 & 0 & 0 \\
                 0 & 0 & \sqrt{2} & 0 & \sqrt{2} & 0 & \sqrt{2} & 0 & 0 \\
                 0 & 0 & 0 & 0 & 0 & \sqrt{2} & 0 & \sqrt{2} & 0 \\
                 1 & 0 & 0 & 1 & 0 & 0 & 1 & 0 & 0 \\
                 0 & 1 & 0 & 0 & 1 & 0 & 0 & 1 & 0 \\
                 0 & 0 & 1 & 0 & 0 & 1 & 0 & 0 & 1 \\
                 0 & 0 & 0 & \sqrt{2} & 0 & 0 & 0 & \sqrt{2} & 0 \\
                 \sqrt{2} & 0 & 0 & 0 & \sqrt{2} & 0 & 0 & 0 & \sqrt{2} \\
                 0 & \sqrt{2} & 0 & 0 & 0 & \sqrt{2} & 0 & 0 & 0
               \end{pmatrix*}.
\end{align}
\noindent Aus den gemessenen Intensitäten für die verschiedenen Richtungen
lassen sich die Absorptionskoeffizienten bestimmen. Würfel $1$ dient dabei
als Referenzmessung und besteht selbst lediglich aus dem Aluminiumblech, welches
die Hülle bildet. Deren Einfluss auf die Absorption wird daher bei allen
Adsorptionskoeffizienten berücksichtigt und herausgerechnet. Gemäß den
Gleichungen \ref{eqn:03} bis Gleichung \ref{eqn:05} ergeben sich somit die
Absorptionskoeffizienten der jeweiligen Würfel unter Berücksichtigung ihrer
Messunsicherheiten zu
\begin{align}
  \vec{\mu}_2 = \begin{pmatrix*}[c]
                0.06443602 \pm 0.00409104 \\
                0.04719267 \pm 0.00302454 \\
                0.04687276 \pm 0.00407686 \\
                0.03838809 \pm 0.0030178 \\
                0.04866847 \pm 0.00318385 \\
                0.07343893 \pm 0.00303886 \\
                0.04562751 \pm 0.00409217 \\
                0.06043446 \pm 0.0030291 \\
                0.04096713 \pm 0.00407606
                \end{pmatrix*},
\end{align}
\begin{align}
  \vec{\mu}_3 = \begin{pmatrix*}[c]
                0.86052077 \pm 0.01313316  \\
                1.28121956 \pm 0.01034112  \\
                0.92677002 \pm 0.01386284  \\
                0.83604375 \pm 0.00939364  \\
                0.95146769 \pm 0.01101726  \\
                1.30528224 \pm 0.01066586 \\
                1.02353279 \pm 0.01302184  \\
                0.98951325 \pm 0.00955678  \\
                1.03059674 \pm 0.01378654
                \end{pmatrix*}
\end{align}
\noindent und
\begin{align}
  \vec{\mu}_4 = \begin{pmatrix*}[c]
                 0.12720674   \pm 0.00574798  \\
                 0.98520605   \pm 0.00501731  \\
                 0.10406502   \pm 0.00578873  \\
                 -0.05209837  \pm  0.00453632 \\
                 1.03778897   \pm 0.00530894  \\
                 0.20942458   \pm 0.0047592   \\
                 0.16830403   \pm 0.00571699  \\
                 0.94291212   \pm 0.00495178  \\
                 0.10017016   \pm 0.0058061
                \end{pmatrix*}.
\end{align}
\noindent Die inneren Elementarwürfel der Würfel $2$ und $3$ bestehen jeweils
aus den gleichen Elementen, daher können Ihre Absorptionskoeffizienten über die
verschiedenen Einträge der Vektoren $\mu_i$ für $i = 1, 2, 3, 4$ gemittelt
werden. Für die gemittelten Absorptionskoeffizienten gilt
\begin{align}
  \mu_2 = (0.051780670154972314  \pm  0.011020710397047466) \: \si{\centi\metre^{-2}}\\
  \mu_3 = (1.0227718680457842  \pm  0.15750439584823495) \: \si{\centi\metre^{-2}}.
\end{align}
\noindent Für Würfel $4$ gilt dies nicht, er besteht aus verschiedenen
Elementarwürfeln, welche aus den in Würfel $2$ und $3$ verarbeiteten Materialien
bestehen.
