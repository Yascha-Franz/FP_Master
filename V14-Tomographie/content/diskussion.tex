\noindent Um die Messergebnisse interpretieren zu können, müssen diese mit den
Literaturwerten verglichen werden. Basisdaten hierzu werden vom nationalen
Institut für Standards und Technologien \cite{nist} bereitgestellt. Die
möglichen Materialien sind Messing CuZn, Delerin CH${_2O}$, Blei $^{82}$Pb,
Aluminium $^{13}$Al und
Eisen $^{26}$Fe. Die zugehörigen Absorptionskoeffizienten $\mu$ berechnen sich
auch dem Produkt des Massenschwächungskoeffizienten und der Dichte. Die
Literaturwerte und berechneten Absorptionskoeffizienten sind Tabelle \ref{tab:01}
zu entnehmen.
\begin{table}
  \centering
  \caption{Materialabhängige Absorptionskoeffizienten \cite{nist} \cite{density}}
  \label{tab:01}
  \begin{tabular}{c c c c}
    \toprule
   \multicolumn{1}{c}{Material} & \multicolumn{1}{c}{Massenabschwächungskoeffizient}
   & \multicolumn{1}{c}{Dichte} & \multicolumn{1}{c}{Absorptionskoeffizient} \\
   \midrule
    Aluminium &\SI{0.078}{\square\centi\metre\per\gram}&\SI{2.699}{\gram\per\cubic\centi\metre} &\SI{0.211}{\centi\metre^{-2}} \\
    Blei      &\SI{0.125}{\square\centi\metre\per\gram}&\SI{11.35}{\gram\per\cubic\centi\metre}& \SI{1.419}{\centi\metre^{-2}} \\
    Eisen     &\SI{0.077}{\square\centi\metre\per\gram}&\SI{7.874}{\gram\per\cubic\centi\metre} &\SI{0.606}{\centi\metre^{-2}} \\
    Messing   &\SI{0.077}{\square\centi\metre\per\gram}&\SI{8.284}{\gram\per\cubic\centi\metre}& \SI{0.638}{\centi\metre^{-2}} \\
    Delerin   &\SI{0.086}{\square\centi\metre\per\gram}&\SI{1.405}{\gram\per\cubic\centi\metre}& \SI{0.121}{\centi\metre^{-2}} \\
\bottomrule
  \end{tabular}
\end{table}
\noindent Die für den Würfel $2$ berechneten Absorptionkoeffizienten stimmen nicht
gut mit den Literaturwerten für die verschiedenen Materialien überein. Wenngleich
dieser etwa der Hälfte des Absorptionskoeffizienten das Delerin entspricht, deuten
die Absorptionskoeffizienten der Elementarwürfel in Würfel $4$ an, dass die
Absorptionskoeffizeinten für Delerin und die anderen Elemente wesentlich genauer
bestimmt werden können. Bestätigt wird diese Vermututung auch dadurch, dass der
gemittelte Absorptionskoeffizient für Würfel $3$ mit einer Messunsicherheit von
etwa $\SI{39}{\percent}$ vom Literaturwert für Blei abweicht, während die
Abweichung des Würfels $2$ zum Delerin bei $\SI{132}{\percent}$ und zu Aluminium
bei etwa $\SI{405}{\percent}$ liegt.  \\
\noindent Für den Würfel $4$ lassen sich die Elementarwürfel deutlicher
verschiedenen Materialien zuordnen. Die Zuordnung ist in Tabelle \ref{tab:02}
dargestellt.
\FloatBarrier
\begin{table}
  \centering
  \caption{Absorptionskoeffizienten und Materialübereinstimmungen des Würfels $4$}
  \label{tab:02}
  \begin{tabular}{c c c}
    \toprule
   \multicolumn{1}{c}{$\mu_i$} & \multicolumn{1}{c}{Materialübereinstimmung }
   & \multicolumn{1}{c}{Abweichung zu \ref{tab:01}}  \\
   \midrule
    $\mu_1$ & Delerin &   $\SI{5}{\percent} $\\
    $\mu_2$ & Blei &      $\SI{31}{\percent}$ \\
    $\mu_3$ & Delerin &   $\SI{14}{\percent}$ \\
    $\mu_4$ & Delerin &   $\SI{41}{\percent}$ \\
    $\mu_5$ & Blei &      $\SI{27}{\percent}$ \\
    $\mu_6$ & Aluminium & $\SI{1}{\percent} $\\
    $\mu_7$ & Aluminium & $\SI{21}{\percent}$ \\
    $\mu_8$ & Messing &   $\SI{47}{\percent}$ \\
    $\mu_9$ & Delerin &   $\SI{18}{\percent}$ \\
\bottomrule
  \end{tabular}
\end{table}
\newpage
\FloatBarrier
\noindent Der Versuch zur Tomographie zeigt damit, dass es möglich ist,
verschiedene Strukturen zu unterscheiden. Die Messunsicherheiten deuten allerdings
an, dass eine genaue Unterscheidung nicht immer zuverlässig möglich ist. Dies
liegt zum Teil daran, dass die Materialien nicht gemäß der idealisierten
Richtungszuweisung (vgl. \ref{fig:04}) vom Strahlengang durchleuchtet wurden.
Zusätzlich ist unter Berücksichtigung der Größe der einzelnen Elementarwürfel
die Annahme einer punktförmigen Strahlungsquelle aufzugeben und stattdessen
von einem durchweg kollimierten Strahl ohne transversale Ausdehnung auszugehen.
